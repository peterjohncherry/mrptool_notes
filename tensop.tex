\documentclass[12pt]{article}
\usepackage[utf8x]{inputenc}
\usepackage[english]{babel}
\usepackage[T1]{fontenc}
\usepackage{color}
\usepackage{amsmath}
\usepackage{amssymb}
\usepackage{textcomp}
\usepackage{array}
\usepackage{booktabs}
\usepackage[font=small,format=plain,labelfont=bf,up,textfont=it,up]{caption}
\usepackage{longtable}
\usepackage{calc}
\usepackage{setspace}
\usepackage{hhline}
\usepackage{ifthen}
\usepackage{lscape}
\usepackage{pdfpages}
\usepackage{geometry}
 \geometry{
 a4paper,
 total={170mm,257mm},
 left=20mm,
 top=20mm,
 bottom=20mm,
 right=20mm,
 }
\usepackage{cite}
\linespread{1.3}

\begin{document}
\section{Tensor Operator (TensOp)}
In most cases the ranges of the molecular orbital indexes can be split into a
number of different subranges, which enable us to define each of the operator
tensors as the sum of a number of different blocks, each one associated with a
different combination of these subranges. Decomposing the tensors in this
manner reduces the maximum size of the data structure the computer has to deal
with at any one time. Block decomposition also facilitates the implementation
of symmetry, as many blocks are either equivalent to one another or connected
via some simple transformation. This is particularly relevant in the
relativistic case, where Kramers symmetry can potentially be used to ensure
that the size of the blocks we deal with is no larger than those encountered in
the non-relativistic case.\\

\noindent Before proceeding further it is worth mentioning that the design of
the basic tensor operator object is heavily informed by the functionality
required of the algebraic manipulation routines. For these manipulation 
routines to make sense it is necessary to be very precise about the 
definition of the tensors. Consequently, some of the exposition will 
seem tedious, and the design choices unnatural, at 
lest until the algorithms in the main algebraic
manipulator are discussed in due course.\\ 

\noindent The design of this object is best explained by taking an example of a
specific operator; $\hat{T}$, which may be written
\begin{equation}
\hat{T} =  \sum_{ij}^{\{R_{v} \cup R_{a}\}}\sum_{kl}^{\{R_{c}\cup R_{a}\}} \hat{a}^{\dagger}_{i} \hat{a}^{\dagger}_{j} \hat{a}_{k} \hat{a}_{l} T_{ijkl},
\label{eqn:T_def}
\end{equation}
\noindent where $r_{c}$, $r_{a}$ and $r_{v}$ are disjoint sets of molecular
orbital indexes ( hereafter referred to as ranges).  The coefficients
$T_{ijkl}$, can be thought of as elements of a tensor $\mathbf{T}$, which is
the representation of the operator $\hat{T}$ in the molecular orbital basis.
I shall refer to such tensors as operator tensors. \\

\noindent The summations in  (\ref{eqn:T_def}) can be split up, e.g.,
\begin{equation*}
\hat{T} =  \sum_{i}^{R_{v}}\sum_{j}^{R_{v}}\sum_{k}^{R_{c}}\sum_{l}^{R_{c}}\hat{a}^{\dagger}_{i} \hat{a}^{\dagger}_{j} \hat{a}_{k} \hat{a}_{l} T_{ijkl}
\end{equation*}
\begin{equation*}
+  \sum_{i}^{R_{v}}\sum_{j}^{R_{v}}\sum_{k}^{R_{c}}\sum_{l}^{R_{a}}\hat{a}^{\dagger}_{i} \hat{a}^{\dagger}_{j} \hat{a}_{k} \hat{a}_{l} T_{ijkl}
\end{equation*}
\begin{equation*}
+ \sum_{i}^{R_{v}}\sum_{j}^{R_{v}}\sum_{k}^{R_{a}}\sum_{l}^{R_{c}}\hat{a}^{\dagger}_{i} \hat{a}^{\dagger}_{j} \hat{a}_{k} \hat{a}_{l} T_{ijkl}
\end{equation*}
\begin{equation*}
+ \sum_{i}^{R_{v}}\sum_{j}^{R_{v}}\sum_{k}^{R_{a}}\sum_{l}^{R_{a}}\hat{a}^{\dagger}_{i} \hat{a}^{\dagger}_{j} \hat{a}_{k} \hat{a}_{l} T_{ijkl}
\end{equation*}
\begin{equation*}
+ \sum_{i}^{R_{v}}\sum_{j}^{R_{a}}\sum_{k}^{R_{c}}\sum_{l}^{R_{c}}\hat{a}^{\dagger}_{i} \hat{a}^{\dagger}_{j} \hat{a}_{k} \hat{a}_{l} T_{ijkl}
\end{equation*}
\begin{equation*}
+ .... 
\end{equation*}
\noindent  Alternatively, we could write this as
\begin{equation*}
\hat{T} =
 \sum_{r_{1}}^{\{R_{v},R_{a}\}} \sum_{r_{2}}^{\{R_{v},R_{a}\}} \sum_{r_{3}}^{\{R_{a},R_{c}\}} \sum^{\{R_{a},R_{c}\}}_{r_{4}}
 \sum_{t_{1}}^{r_{1}}\sum_{t_{2}}^{r_{2}}\sum_{t_{3}}^{r_{3}}\sum_{t_{4}}^{r_{4}}
\hat{a}^{\dagger}_{t_{1}} \hat{a}^{\dagger}_{t_{2}} \hat{a}_{t_{3}} \hat{a}_{t_{4}} T^{r_{1}r_{2}r_{3}r_{4}}_{t_{1}t_{2}t_{3}t_{4}}
\label{eqn::block_T} 
\end{equation*}
\noindent where $T^{r_{1}r_{2}r_{3}r_{4}}_{t_{1}t_{2}t_{3}t_{4}}$ are the elements of $\mathbf{T^{r_{1}r_{2}r_{3}r_{4}}}$, which is some
sub block of $\mathbf{T}$ defined by the ranges $r_{1}$, $r_{2}$,
$r_{3}$, and $r_{4}$. It is in the manner outlined in (\ref{eqn::block_T}) that all terms are evaluated in the program.\\

\noindent Accordingly, for each operator $\hat{T}$, we store a single \emph{TensOp} object, which corresponds
to the molecular orbital tensor $\mathbf{T}$.  Each such TensOp object contains a list of \emph{CtrTensorPart}
objects, which correspond to the blocks $\mathbf{T^{r_{1}r_{2}r_{3}r_{4}}}$. The program is
written so that all operations involving $\hat{T}$ are translated into operators involving the \emph{CtrTensorPart}s.

\end{document}
