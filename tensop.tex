\documentclass[12pt]{article}
\usepackage[utf8x]{inputenc}
\usepackage[english]{babel}
\usepackage[T1]{fontenc}
\usepackage{color}
\usepackage{amsmath}
\usepackage{amssymb}
\usepackage{textcomp}
\usepackage{array}
\usepackage{booktabs}
\usepackage[font=small,format=plain,labelfont=bf,up,textfont=it,up]{caption}
\usepackage{longtable}
\usepackage{calc}
\usepackage{setspace}
\usepackage{hhline}
\usepackage{ifthen}
\usepackage{lscape}
\usepackage{pdfpages}
\usepackage{geometry}
 \geometry{
 a4paper,
 total={170mm,257mm},
 left=20mm,
 top=20mm,
 bottom=20mm,
 right=20mm,
 }
\usepackage{cite}
\linespread{1.3}

\begin{document}
\section{Tensor Operator (TensOp)}
In most cases the ranges of the molecular orbital indexes can be split into a
number of different subranges, which enable us to define each of the operator
tensors as the sum of a number of different blocks, each one associated with a
different combination of these subranges. Decomposing the tensors in this
manner reduces the maximum size of the data structure the computer has to deal
with at any one time. Block decomposition also facilitates the implementation
of symmetry, as many blocks are either equivalent to one another or connected
via some simple transformation. This is particularly relevant in the
relativistic case, where Kramers symmetry can potentially be used to ensure
that the size of the blocks we deal with is no larger than those encountered in
the non-relativistic case.\\

\noindent Before proceeding further it is worth mentioning that the design of
the basic tensor operator object is heavily informed by the functionality
required of the algebraic manipulation routines. For these manipulation 
routines to make sense it is necessary to be very precise about the 
definition of the tensors. Consequently, some of the exposition will 
seem tedious, and the design choices unnatural, at least until the algorithms in the main algebraic
manipulator are discussed.\\ 

\noindent The design of this object is easiest explained by example. Consider an operator; $\hat{Q}$,
which may be written in second quantized form as
\begin{equation}
\hat{Q} =  \sum_{i}^{R^{i}}\sum_{j}^{R^{j}}\sum_{k}^{R^{k}}\sum_{l}^{R^{l}} \hat{a}^{\dagger}_{i} \hat{a}^{\dagger}_{j} \hat{a}_{k} \hat{a}_{l} Q_{ijkl}.
\label{eqn:Q_def}
\end{equation}
\noindent  The coefficients,
$Q_{ijkl}$, can be thought of as elements of a tensor $\mathbf{Q}$, which is
the representation of the operator $\hat{Q}$ in the molecular orbital basis.
I shall refer to such tensors as operator tensors. The $R^{i}$ is just the set
of values over which index $i$ runs (note the range is index specific). This
$R^{i}$ can be broken down into $N_{block}$ disjoint subsets or sub ranges, i.e.,
\begin{equation}
R^{i} = r_{1}^{i} \cup r_{2}^{i} \cup r_{3}^{i} \cup ... = \bigcup^{N_{block}}_{\mu} r_{\mu}^{i}
\end{equation}
\noindent A range block, $\mathbf{B}$, defined by a set of subsets;
\begin{equation}
B^{\mu\nu\xi\chi} =  \{ \{i,j,k,l\} | i \in r_{\mu}^{i} , j \in r_{\nu}^{j},  k \in r_{\xi}^{k}, l\in r_{\chi} \}
\end{equation}
and the components, $ Q_{ijkl}^{b}$, of a tensor block, $\mathbf{Q}^{b}$ are those blocks which satisfy
\begin{equation}
Q_{ijkl} \in Q^{b} \text{ \ \ iff \ \ }  \{ i, j, k, l \} \in b   
\end{equation}
using this notation we can rewrite (\ref{eqn:Q_def}) as 
\begin{equation}
\hat{Q} =  \sum_{b}^{\{B\}} \sum_{\{i,j,k,l\}}^{b }
\hat{a}^{\dagger}_{i} \hat{a}^{\dagger}_{j} \hat{a}_{k} \hat{a}_{l} Q_{ijkl}.
\label{eqn:Q_block_def}
\end{equation}
\noindent 

//It is in the manner outlined in (\ref{eqn::block_T}) that all terms are evaluated in the program.\\

\noindent Accordingly, for each operator $\hat{T}$, we store a single \emph{TensOp} object, which corresponds
to the molecular orbital tensor $\mathbf{T}$.  Each such TensOp object contains a list of \emph{CtrTensorPart}
objects, which correspond to the blocks $\mathbf{T^{r_{1}r_{2}r_{3}r_{4}}}$. The program is
written so that all operations involving $\hat{T}$ are translated into operators involving the \emph{CtrTensorPart}s.

\subsection { Block Symmetry }

Some blocks of the tensor can be obtained by transforming other blocks.


\end{document}
