To better illustrate the capabilities of the program below is an example input
for calculating the T-amplitudes in an XMS-CASPT2 :
\definecolor{light-gray}{gray}{0.75}
\lstset{ frameround=fttt, language=Java, basicstyle=\tiny, backgroundcolor=\color{lightgray}}
\begin{lstlisting}[label={lst:input}]
{
  "title" : "proptool",
  "nfrozencore" : 0,
  "nfrozenvirt" : 0,
  "nact" : 4,
  "nclosed" : 3,

  "variables" : {
     "factors" : [ { "name" : "shift", "value" :  0.2 } ],
     "ranges" :  [ { "name" : "states", "range_vec" : [ 0, 1, 2 ] } ,
                   { "name" : "act" ,   "range_vec" : [ 3, 4, 5, 6 ] } ]
   },

  "operators" : [
    { "name"   : "T", "factor" : 1.0, "TimeSymm" : "none", "HermConj" : 0,
      "idxs"   : ["T0", "T1", "T2", "T3"],
      "ranges" : [["act", "vir"], ["act", "vir"], ["cor", "act"], ["cor", "act"]],
      "aops"   : [ 1, 1, 0, 0 ] ,
      "state dependence" : 2 } ] ,

  "terms" : [
    { "name" : "X(H-E-Es)T", "type" : "orb_exc_deriv", "Target" : "T'",
        "brakets" : [
          { "ops"    : [ { "name" : "T'",   "ids" : [ "none" ] },
                         { "name" : "H-E", "ids" : [ "N" ] },
                         { "name" : "T",   "ids" : [ "L", "N" ] } ] ,
                           "bra" : "M", "ket" : "N" }, 
          { "ops"    : [ { "name" : "T'" , "ids" : [ "none" ] },
                         { "name" : "T" , "ids" : [ "L", "N" ] } ] ,
                           "factor" : "shift", "bra" : "M", "ket" : "N" } ] },
   { "name" : "-X(H-E)X", "type" : "orb_exc_deriv",
     "brakets" : [
       { "ops"    : [ { "name" : "X",   "ids" : [ "none" ] },
                      { "name" : "H-E", "ids" : [ "N" ] },
                      { "name" : "X",   "ids" : [ "none" ] } ] ,
                        "bra" : "M", "ket" : "M" } ] } ],

  "equations" :  [
     { "name" : "T_update", "type" : "LinearRM" , "target" : "T", "target indexes" : [ "L", "N" ],
      "expressions" : [
         { "name"  : "residual",
           "terms" : [
             { "term" : "X(H-E-Es)T", "factor" : "one",
               "indexes" : [ { "name" : "M", "range" : "states", "sum" : false },
                             { "name" : "L", "range" : "states", "sum" : false },
                             { "name" : "N", "range" : "states", "sum" : false } ] } ] }
         { "name"  : "denominator",
           "terms" : [
             { "term" : "X(H-E)X", "factor" : "one",                                                 
               "indexes" : [ { "name" : "M", "range" : "states", "sum" : false },
                             { "name" : "L", "range" : "states", "sum" : false },
                             { "name" : "N", "range" : "states", "sum" : false } ] } ] } ]
   } ]
}                                                 
\end{lstlisting}         

\noindent \textbf{Important Note}: As the program is still undergoing development and debugging, 
the input structure is highly volatile, and the above is unlikely to be compatible with
the current version of the program. However, the basic contents and guiding principals
are unlikely to change substantially.\\

\noindent The input makes use of the json format discussed more fully in  \cite{JSON}.
Interface is currently rather prolix, and will be substantially reduced in complexity
as the program nears full release. Nonetheless, the above form is useful for demonstrating the capabilities
of the program. \\

\noindent To explain the above it is best to discuss each section one at a time. First,
\lstset{ frameround=fttt, language=Java, basicstyle=\tiny, backgroundcolor=\color{lightgray}}
\begin{lstlisting}[label={lst:casscf}]
  "frozen" : false,
  "nfrozenvirt" : 0,
  "nact" : 4,
  "nclosed" : 3,
\end{lstlisting}         
\noindent These keywords just designate the number of the various kinds of orbitals used in the CASSCF.
The "\texttt{nfrozencore} and \texttt{nfrozenvirt} specify the number of core or virtual orderings to be "frozen";
such orbitals are not excited either from or to.\\

\lstset{ frameround=fttt, language=Java, basicstyle=\tiny, backgroundcolor=\color{lightgray}}
\begin{lstlisting}[label={lst:variables}]
  "variables" : {
     "factors" : [ { "name" : "shift", "value" :  0.2 } ],
     "ranges" :  [ { "name" : "states", "range_vec" : [ 0, 1, 2 ] } ,
                   { "name" : "act" ,   "range_vec" : [ 3, 4, 5, 6 ] } ,
                   { "name" : "virt" ,  "range_vec" : [ 7, 8, 9, 10, 11, 12, 13, 14, 15 ] } ]
   },
\end{lstlisting}         
\noindent This section is used to define ranges and variables which may be used in later definitions. For example,
when \texttt{shift} is specified here to be 0.2 (default units are Hartrees, and currently, this cannot be 
changed). 

\lstset{ frameround=fttt, language=Java, basicstyle=\tiny, backgroundcolor=\color{lightgray}}
\begin{lstlisting}[label={lst:operators}]
  "operators" : [
    { "name"     : "T",
      "factor" : 1.0,
      "TimeSymm" : "none",
      "HermConj" : true,
      "idxs"     : ["T0", "T1", "T2", "T3"],
      "ranges"   : [["act", "vir"], ["act", "vir"], ["cor", "act"], ["cor", "act"]],
      "aops"     : [ 1, 1, 0, 0 ] ,
      "state dependence" : 2 } ] ,
\end{lstlisting}         
\noindent This section enables the user to define their own operator, and
specify some of its symmetries. For example, here \texttt{TimeSymm} and
\texttt{HermConj} determine whether this operator has time reversal symmetry 
and whether or not it is Hermitian. The symmetry functions are 
currently hard coded, but there is the facility for user defined ones
to be added with relative ease.
 The \texttt{aops} specifies the ordering of the   
creation and annihilation operators; 1 is creation, 0 is annihilation.  Here it
is specified that the T tensor has four indices. The \texttt{ranges} array specifies the
orbitals on which the operator acts.  Each subarray in the \texttt{ranges} section
specifies a different sub range used to divide up the T tensor into blocks.
For example, in one possible block of the T operator the first index would be
active, the second index could be virtual, whilst the third and fourth blocks
would be core.  Note that in the aforementioned \texttt{variables} section the user is
able to specify the exact orbital indexes to which each range corresponds,
enabling the user to have very tight control over the block decomposition of
the tensor, and the ranges on which an operator acts. 
The \texttt{state dependence} section determines how many state indexes 
the operator has.
\lstset{ frameround=fttt, language=Java, basicstyle=\tiny, backgroundcolor=\color{lightgray}}
\begin{lstlisting}[label={lst:terms}]
  "terms" : [
    { "name" : "residual", "type" : "orb_exc_deriv", "Target" : "T'",
        "brakets" : [
          { "ops"    : [ { "name" : "T'",   "ids" : [ "none" ], "transform" : "inverse" },
                         { "name" : "H-E", "ids" : [ "L" ] },
                         { "name" : "T",   "ids" : [ "L", "N" ] } ] ,
                           "bra" : "M", "ket" : "N" }, 

          { "ops"    : [ { "name" : "T'" , "ids" : [ "none" ], "transform" : "inverse" },
                         { "name" : "T" , "ids" : [ "L", "N" ] } ] ,
                           "factor" : "shift", "bra" : "M", "ket" : "N" } ] },

   { "name" : "-X(H-E)X", "type" : "value", "target" : "T'",
     "brakets" : [
       { "ops"    : [ { "name" : "X",   "ids" : [ "none" ] },
                      { "name" : "H-E", "ids" : [ "L" ] },
                      { "name" : "X",   "ids" : [ "none" ] } ] ,
                        "bra" : "N", "ket" : "N" } ] } ],
\end{lstlisting}         
The terms section defines quantities to facilitate construct the expressions
the user wishes the program to evaluate.  The \texttt{name} is used by the program to
identify the expression internally, and in printing the results. It must be unique. 
The \texttt{type : orb\_exc\_deriv} specifies the kind of term;  in this case the partial
derivative of the terms is taken with respect to the elements of the tensor specified by \texttt{Target : T'}.\\

\noindent To clarify things, it is worth considering the \texttt{brakets} array seperately:

\lstset{ frameround=fttt, language=Java, basicstyle=\tiny, backgroundcolor=\color{lightgray}}
\begin{lstlisting}[label={lst:brakets}]
"brakets" : [
  { "ops"    : [ { "name" : "X",   "ids" : [ "none" ] },
                 { "name" : "H-E", "ids" : [ "L" ] },
                 { "name" : "T",   "ids" : [ "L", "N" ] } ] ,
                   "bra" : "M",
                   "ket" : "N"                          }, 
  { "ops"    : [ { "name" : "X" , "ids" : [ "none" ] },
                 { "name" : "T" , "ids" : [ "L", "N" ] } ] ,
                   "factor" : "shift",
                   "bra" : "M",
                   "ket" : "N" } ] 
\end{lstlisting}         
The above represents the mathematical term:
\begin{equation}
\sum_{T'}
\langle \Psi_{M} |  \hat{X}_{ijkl} ( \hat{H}-E^{0}_{L} ) \hat{T}^{LN}_{ t_{0}t_{1}t_{2}t_{3} } | \Psi_{N} \rangle + 
E_{shift}\langle \Psi_{M} |  \hat{X}_{ijkl}  \hat{T}^{LN}_{ t_{0}t_{1}t_{2}t_{3} }| \Psi_{N} \rangle 
\end{equation}
\noindent Note that there are no summations specified in the \texttt{brakets} array. 
The \texttt{ops} specifies the operators used, and \texttt{ids} specifies the names of the state indexes
for these operators. It is important to note that several operators, or combinations of
operators with scalar arrays, e.g., "H-E", are predefined in the code. 
This is necessary for defining the ranges of summations later on, and also to enable indexes on 
different operators to be tied to one another. For example the first index, $L$, on the T tensor is the same in 
both brakets. The \texttt{none} indicates that the operator is state independent.
The \texttt{factor} defines the factor associated by which the braket is multiplied, and this
defaults to 1.0. The \texttt{Bra} and \texttt{Ket} specify the name of the state index of the Bra and Ket. \\

\lstset{ frameround=fttt, language=Java, basicstyle=\tiny, backgroundcolor=\color{lightgray}}
\begin{lstlisting}[label={lst:equations}]
"equations" :  [
   { "name" : "T_update", "type" : "LinearRM" , "target" : "T", "target indexes" : [ "L", "N" ],
    "expressions" : [
       { "name"  : "residual",
         "terms" : [
           { "term" : "residual", "factor" : "mone",
             "indexes" : [ { "name" : "M", "range" : "states", "sum" : false },
                           { "name" : "L", "range" : "states", "sum" : false },
                           { "name" : "N", "range" : "states", "sum" : true } ] },
       { "name"  : "denominator",
         "terms" : [
           { "term" : "X(H-E)X", "factor" : "one",                                                 
             "indexes" : [ { "name" : "M", "range" : "states", "sum" : false },
                           { "name" : "L", "range" : "states", "sum" : false },
                           { "name" : "N", "range" : "states", "sum" : true } ] } ] } ]
    } ]
  } ]
\end{lstlisting}         
The \texttt{equations} section is used to specify what is actually calculated. The above corresponds to
finding the \texttt{T} which minimizes the following
\begin{equation}
\frac{ \Phi_{M} |  \hat{X}_{\Omega} ( \hat{H}-E^{0}_{L}-E_{shift} )  \hat{T}^{LN}_{\Omega'} | \Phi_{N} \rangle } 
{ \langle \Phi_{N}   | \hat{X}^{\dagger}_{\Omega}
                     ( \hat{H}- E^{0}_{L} )
                       \hat{X}_{\Omega'} |
 \Phi_{M} \rangle}
\end{equation}
It should be noted that for reasons of clarity, the various transformations and orthogonalizations associated
with XMS-CASPT2 and the internally contracted wavefunctions are not included here. Ultimately, XMS-CASPT2
will be assumed to be the default, and the corresponding transformations will be performed automatically, however
this will not substantially impact the input. \\

\noindent The  \texttt{equations} section first specifies the \texttt{type} of equation to be solved, in
this case \texttt{LinearRM} (linear residual minimization). The type dictates what other keywords must appear.
Due to the choice of the \texttt{LinearRM} type, it is necessary to specify a \texttt{target} tensor for 
which we are solving, as well as a \texttt{residual} and \texttt{denominator}. These are two \texttt{expressions}
which are defined by a list of \texttt{term}s, which were themselves defined by the user in the previous section. 
It is at this stage that the user specifies the summations over the state indexes which appear in the terms.

\noindent An expression may be composed of multiple different terms even though, in principle, these could be merged.
However, breaking down the expression into smaller, independent parts facilitates identification of
quantities which may be reused, and eases control of summations over different state indexes. 
se terms with the exception that it is also necessary to specify the ranges over
which the summations on the state indexes are performed.
