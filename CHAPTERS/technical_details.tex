\documentclass[12pt]{article}
\usepackage[utf8x]{inputenc}
\usepackage[english]{babel}
\usepackage[T1]{fontenc}
\usepackage{amsmath}
\usepackage{amssymb}
\usepackage{textcomp}
\usepackage{array}
\usepackage{pdfpages}
\usepackage{geometry}
 \geometry{
 a4paper,
 total={170mm,257mm},
 left=15mm,
 top=15mm,
 bottom=15mm,
 right=15mm,
 }

\usepackage{cite}
\linespread{1.3}

\begin{document}
\section{ More Supplementary material }
\subsection{ Conversion from Bagel/SMITH3 ordering to normal ordering } 

Several of the tensors (notably $\mathbf{v2}$ and $\mathbf{T}$ )  in Bagel and SMITH3 are associated with the operator
ordering:
\begin{equation}
\langle I | \hat{a}_{l} \hat{a}_{j}^{\dagger} \hat{a}_{k} \hat{a}_{i}^{\dagger} | J \rangle X_{ljki},
\end{equation}
where $\mathbf{X}$ is some arbitrary tensor representation.  Using the commutation relations this can be
converted to normal ordering as :
\begin{equation}
- \hat{i}^{\dagger} \hat{j}^{\dagger} \hat{k}\hat{l} 
- \hat{i}^{\dagger} \hat{k} \delta_{jl}
+ \hat{j}^{\dagger} \hat{k} \delta_{il} 
- \hat{j}^{\dagger} \hat{l} \delta_{ik}
+ \delta_{ik}\delta_{jl} )
\end{equation}
where  $ \hat{i}^{\dagger} =\hat{a}_{i}^{\dagger}$. 

\subsubsection{ Case of $\mathbf{T}$ } 
For the CASPT2 amplitudes, $\mathbf{T}$, note that $k$ and $l$  range over closed, $c$, and active, $x$, orbitals, whilst 
$i$ and $j$ range over active, $x$, and virtual, $v$, orbitals. Furthermore, it is subject to the constraint that
at no point may all the indexes be active.

\noindent This means all the indexes on the  $\delta$ are must always active as 
this is the only way the creation and annihilation indexes can have the same range. This 
ensures that regardless of the block, only a single 1-electron density matrix contributes
at any point, and that the final term involving two deltas always vanishes.\\

The remaining cases are:
\begin{itemize} 
\item $r_{i} \text{, \ }r_{j}\text{, \ } r_{k} \text{, \ } r_{l}   \rightarrow \text{ \ \ \ \  non-zero terms }$

\item $xxxc \rightarrow \text{ \ \ \ \ } [- \hat{i}^{\dagger} \hat{j}^{\dagger} \hat{k}\hat{l} - \hat{j}^{\dagger} \hat{l} \delta_{ik}]$

\item $xxcx 
\rightarrow \text{ \ \ \ \ } [- \hat{i}^{\dagger} \hat{j}^{\dagger} \hat{k}\hat{l}
 - \hat{i}^{\dagger} \hat{k} \delta_{jl} + \hat{j}^{\dagger} \hat{k} \delta_{il}]
$

\item $xxcc 
\rightarrow \text{ \ \ \ \ } [- \hat{i}^{\dagger} \hat{j}^{\dagger} \hat{k}\hat{l}] $

\item $xvxx 
\rightarrow \text{ \ \ \ \ } [- \hat{i}^{\dagger} \hat{j}^{\dagger} \hat{k}\hat{l}
+ \hat{j}^{\dagger} \hat{k} \delta_{il} 
- \hat{j}^{\dagger} \hat{l} \delta_{ik}] $

\item $xvxc 
\rightarrow \text{ \ \ \ \ } [- \hat{i}^{\dagger} \hat{j}^{\dagger} \hat{k}\hat{l}
- \hat{j}^{\dagger} \hat{l} \delta_{ik}] $

\item $xvcx 
\rightarrow \text{ \ \ \ \ } [- \hat{i}^{\dagger} \hat{j}^{\dagger} \hat{k}\hat{l}
+ \hat{j}^{\dagger} \hat{k} \delta_{il} ]$

\item $xvcc 
\rightarrow \text{ \ \ \ \ } [- \hat{i}^{\dagger} \hat{j}^{\dagger} \hat{k}\hat{l}]$

\item $vxxx 
\rightarrow \text{ \ \ \ \ } [- \hat{i}^{\dagger} \hat{j}^{\dagger} \hat{k}\hat{l}
- \hat{i}^{\dagger} \hat{k} \delta_{jl}]$

\item $vxxc 
\rightarrow \text{ \ \ \ \ } [- \hat{i}^{\dagger} \hat{j}^{\dagger} \hat{k}\hat{l}]$

\item $vxcx 
\rightarrow \text{ \ \ \ \ } [- \hat{i}^{\dagger} \hat{j}^{\dagger} \hat{k}\hat{l}
- \hat{i}^{\dagger} \hat{k} \delta_{jl}]$

\item $vxcc 
\rightarrow \text{ \ \ \ \ } [- \hat{i}^{\dagger} \hat{j}^{\dagger} \hat{k}\hat{l}]$

\item $vvxx 
\rightarrow \text{ \ \ \ \ } [- \hat{i}^{\dagger} \hat{j}^{\dagger} \hat{k}\hat{l}]$

\item $vvxc
\rightarrow \text{ \ \ \ \ } [- \hat{i}^{\dagger} \hat{j}^{\dagger} \hat{k}\hat{l}]$

\item $vvcx
\rightarrow \text{ \ \ \ \ } [- \hat{i}^{\dagger} \hat{j}^{\dagger} \hat{k}\hat{l}]$

\item $vvcc
\rightarrow \text{ \ \ \ \ } [- \hat{i}^{\dagger} \hat{j}^{\dagger} \hat{k}\hat{l}]$
\end{itemize}
If we assume that there is a summation over all indexes, then several terms will cancel:
\begin{itemize} 
\item $r_{i} \text{, \ }r_{j}\text{, \ } r_{k} \text{, \ } r_{l}   \rightarrow \text{ \ \ \ \  non-zero terms }$

\item $xxxc \rightarrow \text{ \ \ \ \ } -\hat{i}^{\dagger} \hat{j}^{\dagger} \hat{k}\hat{l} - \hat{j}^{\dagger} \hat{l} \delta_{ik}$

\item $xxcx \rightarrow \text{ \ \ \ \ } -\hat{i}^{\dagger} \hat{j}^{\dagger} \hat{k}\hat{l}  $

\item $xvxx \rightarrow \text{ \ \ \ \ } -\hat{i}^{\dagger} \hat{j}^{\dagger} \hat{k}\hat{l}$

\item $vxxx \rightarrow \text{ \ \ \ \ } -\hat{i}^{\dagger} \hat{j}^{\dagger} \hat{k}\hat{l} - \hat{i}^{\dagger} \hat{k} \delta_{jl}$

\item $xvxc \rightarrow \text{ \ \ \ \ } -\hat{i}^{\dagger} \hat{j}^{\dagger} \hat{k}\hat{l} - \hat{j}^{\dagger} \hat{l} \delta_{ik}$

\item $xvcx \rightarrow \text{ \ \ \ \ } -\hat{i}^{\dagger} \hat{j}^{\dagger} \hat{k}\hat{l} + \hat{j}^{\dagger} \hat{k} \delta_{il}$

\item $vxxc \rightarrow \text{ \ \ \ \ } -\hat{i}^{\dagger} \hat{j}^{\dagger} \hat{k}\hat{l}$

\item $vxcx \rightarrow \text{ \ \ \ \ } -\hat{i}^{\dagger} \hat{j}^{\dagger} \hat{k}\hat{l} - \hat{i}^{\dagger} \hat{k} \delta_{jl}$

\item $xvcc \rightarrow \text{ \ \ \ \ } -\hat{i}^{\dagger} \hat{j}^{\dagger} \hat{k}\hat{l}$

\item $vxcc \rightarrow \text{ \ \ \ \ } -\hat{i}^{\dagger} \hat{j}^{\dagger} \hat{k}\hat{l}$

\item $vvxc \rightarrow \text{ \ \ \ \ } -\hat{i}^{\dagger} \hat{j}^{\dagger} \hat{k}\hat{l}$

\item $vvcx \rightarrow \text{ \ \ \ \ } -\hat{i}^{\dagger} \hat{j}^{\dagger} \hat{k}\hat{l}$

\item $vvxx \rightarrow \text{ \ \ \ \ } -\hat{i}^{\dagger} \hat{j}^{\dagger} \hat{k}\hat{l}$

\item $xxcc \rightarrow \text{ \ \ \ \ } -\hat{i}^{\dagger} \hat{j}^{\dagger} \hat{k}\hat{l} $

\item $vvcc \rightarrow \text{ \ \ \ \ } -\hat{i}^{\dagger} \hat{j}^{\dagger} \hat{k}\hat{l}$
\end{itemize}


\subsubsection{ Case of $\mathbf{H}$ } 
The two electron Hamiltonian is not so simple; it has no such handy constraints. However, we should note
that each one of the creation indexes has to be equal to one of the other creation indexes, or else either all
the 1-electron density matrices, or all the delta functions, will vanish.

\begin{itemize}
\item $r_{i} = r_{k} = x \text{ \ and \ } r_{j} = r_{l} = c,v  \rightarrow \text{ \ \ \ \ } 
[- \hat{i}^{\dagger} \hat{j}^{\dagger} \hat{k}\hat{l} - \hat{i}^{\dagger} \hat{k} \delta_{jl}]$\\

\item $r_{i} = r_{k} = x \text{ \ and \ } r_{j} = r_{l} = c,v  \rightarrow \text{ \ \ \ \ } 
[- \hat{i}^{\dagger} \hat{j}^{\dagger} \hat{k}\hat{l} - \hat{i}^{\dagger} \hat{k} \delta_{jl}]$\\

\item $r_{i} = r_{k} = x \text{ \ and \ } r_{j} = r_{l} = c,v  \rightarrow \text{ \ \ \ \ } 
[- \hat{i}^{\dagger} \hat{j}^{\dagger} \hat{k}\hat{l} - \hat{i}^{\dagger} \hat{k} \delta_{jl}]$\\

\item $r_{j} = r_{k} = x \text{ \ and \ } r_{i} = r_{l} = c,v  \rightarrow \text{ \ \ \ \ } 
[- \hat{i}^{\dagger} \hat{j}^{\dagger} \hat{k}\hat{l} + \hat{j}^{\dagger} \hat{k} \delta_{il}]$\\

\item $r_{j} = r_{l} = x \text{ \ and \ } r_{i} = r_{k} = c,v  \rightarrow \text{ \ \ \ \ } 
[- \hat{i}^{\dagger} \hat{j}^{\dagger} \hat{k}\hat{l} + \hat{j}^{\dagger} \hat{k} \delta_{il}]$\\
\end{itemize}

For the CASPT2 amplitudes, $\mathbf{T}$, note that $k$ and $l$  range over closed, $c$, and active, $x$, orbitals, whilst 
$i$ and $j$ range over active, $x$, and virtual, $v$, orbitals. Furthermore, it is subject to the constraint that
at no point may all the indexes be active.\\

\noindent This means all but the first terms vanishes; the indexes on the  $\delta$ are must always active as 
this is the only way the creation and annihilation indexes can have the same range. However, if these
two indexes are active, then the uncontracted indexes cannot both be active, due to the constraint
that not all indexes on $\mathbf{T}$ can be active. Hence, Bra-Ket vanishes, as both $\langle I | $ and
$| J \rangle $ are active determinants. The final $\delta_{ik}\delta_{jl}$ term vanishes by the same logic.\\



\end{document}
