\noindent The introduction section illustrated how the input is broken down to yield a number of smaller and more manageable \emph{terms}, 
generally corresponding to a single braket. Each of these terms is fed into the algebraic manipulator. This generates a set of instructions for
evaluating this term, which are the task translator then converts into calls to the FCI and linear algebra routines. Construction
of this task list is split up into two major parts; the tensor contraction task list generator, and the 
"gamma list" generator. Before these individual components are discussed in detail it is necessary to discuss the qualities 
the final task list should possess, and what distinguishes and motivates the current approach. \\

\section{Core functionality}
\noindent At its core, the basic operation of the program is to evaluate terms such as 
\begin{equation}
\sum_{ijkl}\sum_{mnop} \sum_{IJ} \langle I | i^{\dagger}j^{\dagger}klm^{\dagger}n^{\dagger}op | J \rangle c^{M}_{I} c_{J}^{N} Y^{ML}_{ijkl}Z^{NP}_{mnop},
\label{eqn:basic_term}
\end{equation}
where $I$ and $J$ each denote a Slater determinant, and $L,M,N,P$ are state
indexes.  $\mathbf{Y}^{ML}$ and $\mathbf{Z}^{NP}$ are representations of
operators and $\hat{Y}$ and $\hat{Z}$ in the molecular orbital basis. For the time being
the state dependence of the operator representations will be ignored, but has important consequences
for exploitation of symmetry, and will be discussed at length later. \\ 

\noindent The computational cost of directly evaluating terms such as (\ref{eqn:basic_term}) can be prove prohibitive,
particularly if the ranges of the indexes $i$, $j$, $k$, $l$, $m$, $n$, $o$, and $p$ are large (e.g., if they
range over virtual and core orbitals). The algebraic manipulator uses various commutation and symmetry relations
to rearrange expressions, such as (\ref{eqn:basic_term}), into a sum of terms in which
the orbital indexes only range over the active orbitals, e.g.,
\begin{equation}
=\sum_{stuvwxyz}\gamma_{stuvwxyz} A_{stuvwxyz}
+\sum_{stuvxy} \gamma_{stuvwx} A_{stuvwx}
\end{equation}
\begin{equation}
+\sum_{stuv}\sum_{mnop} \gamma_{stuv} A_{stuv},
+\sum_{st} \gamma_{st}  A_{st}
+  A .
\label{eqn:A_list}
\end{equation}

\noindent Where the tensors $A_{ijkl...}$ are formed by performing the 
contractions between and re-orderings of the indexes of tensors on $Y$ and $Z$ as 
determined from the commutation relations of the creation and annihilation operations, e.g.,
 
\begin{equation*}
A_{stwx} = \sum_{r}^{R}\hat{\wp}_{r}\sum^{c1}_{\{u,y\}}\delta_{uy}\sum^{c2}_{\{v,z\}} \delta_{vz}Y_{ijkl}Z_{mnop},
\end{equation*}

\noindent where the $c1$ and $c2$ are sets of pairs of indexes to be contracted, e.g.,
\begin{equation}
c2 = \{ \{i, k\} \} , \{j, k\} \} ,\{l, m\} \}.....\} 
\end{equation}
and $\hat{\wp}_{r}$ which transforms the ordered set of indexes
$\{u,v,s,t,w,x,y,z\}$ into some permutation, $r\in R$, of the ordered set of
indexes $\{i,j,k,l,m,n,o,p\}$, whilst also acting to multiply the result of the
summation by an appropriate factor.\\

\noindent The $\gamma$ and $\Gamma$ are defined by\footnote{It is worth noting that these are not normal ordered. The significance
of this will be explained in due course.}:
\begin{equation}
\gamma_{swtzuyvx} = \sum_{IJ} \langle I | s^{\dagger}wt^{\dagger}zu^{\dagger}yv^{\dagger}x | J \rangle c_{I} c^{*}_{J},
\end{equation}
\begin{equation}
\Gamma_{swtzuyvx}^{I} = 
\sum_{stuvwxyz} \sum_{J} \langle I | s^{\dagger}wt^{\dagger}zu^{\dagger}yv^{\dagger}x | J \rangle c^{*}_{J},
\end{equation}

\noindent A key feature of the program is that it is able to use decompose the $\mathbf{A}$ and $\mathbf{\gamma}$ objects
up into smaller sub-blocks, and use symmetries existing between these blocks, as well as number of other physical constraints
on the possibility of creation and annhilation operations, to inform the generation of the task list. Not only does this 
facilitate parallelization of the resulting task list, as operations concerning different blocks can be assigned to 
different nodes, but it may also substantially reduce the number of distinct operations which need to be performed. 
This kind of decomposition is particularly advantageous for relativistic systems, where the seperate treatment of
$\alpha$ and $\beta$ electrons can result in much larger tensors. For example, if the range of each index on
an eight index tensor is doubled, then the number of elements increases by a factor of 256.\\ 

\noindent The challenge faced by the algebraic manipulator is that whilst there are several different
series of form (\ref{eqn:A_list}) which are equal to (\ref{eqn:basic_term}), these different series 
often differ wildly in the amount of computational effort their evaluation requires. Hence the algebraic manipulator must
accomplish two things. First, it must decide on the most series of form (\ref{eqn:A_list}) that is least computationally expensive to
evaluate. Second, it must determine how to evaluate the terms in this series. This second point may seem trivial,
but is complicated substantially by the decomposition of the tensors, and the need to take advantage of the symmetry in 
an effective manner\footnote{Whilst
the algebraic manipulator can also generate task lists for derivative expressions, the task lists remain 
confined to evaluation of linear combinations of terms with one of the following forms : 

\begin{equation}
\sum_{\substack{ijkl \\ mnop} }\gamma_{ijklmnop} A_{ijklmnop},
\text{ \ \ \ \ \ \ \ \ }
\sum_{ijkl}\gamma_{imjo} A_{imjoknlp},
\text{ \ \ \ \ or  \ \ \ \ }
\sum_{\substack{ijkl \\ mnop}}\Gamma^{I}_{ijklmnop} A_{ijklmnop}.
\label{eqn:kinds_of_terms}
\end{equation}
\noindent and these distinctions do not impact the initial discussion.}.\\

\noindent The algebraic manipulator accomplishes this by performing the operations as outlined in figure
% \ref{fig:alg_manip_overview}.

\noindent Chapter 5 focuses on the machinery (called the "gamma generator") for chosing the form of the series (\ref{eqn:A_list}),
whilst Chapter 4 focuses on evaluation of terms contained therein. The reason for this seemingly backwards order is that
the decisions made in the former and wholly dependent on the factors governing the 
efficiency of the task lists generated in the latter.\\

However, we want blocks; 
1) Identify unique tensor blocks.

2) Feed each block into gamma task list generator -> generates A and gammas block list.

3) Merge all gamma task lists generated.

3) Construst A task lists for all gamma blocks.

4) Begin loop over gamma blocks. 

5) Identify required mo tensor blocks. 

6) Execute A task list to obtain all A associated with a given gamma block.

7) Calculate blocks of gammas. 

8) Contract gamma block  with A and save result.

9) Throw away information which will not be needed by later iterations. 

10) Cycle loop until out of gamma blocks.

11) Combine into final result.



