Consider a term
\begin{equation}
\sum_{ijkl}\sum_{mnop} \sum_{IJ} \langle I | i^{\dagger}j^{\dagger}klm^{\dagger}n^{\dagger}op | J \rangle c^{M}_{I} c_{J}^{N} Y^{ML}_{ijkl}Z^{NP}_{mnop},
\label{eqn:basic_term}
\end{equation}
where $I$ and $J$ each denote a Slater determinant, and $L,M,N,P$ are state
indexes.  $\mathbf{Y}^{ML}$ and $\mathbf{Z}^{NP}$ are representations of
operators and $\hat{Y}$ and $\hat{Z}$ in the molecular orbital basis. For the time being
the state dependence of the operator representations will be ignored, but has important consequences
for exploitation of symmetry, and will be discussed at length later. \\ 

\noindent The procedure most commonly employed by code generators is to use Wick's theorem
to rearrange expressions, such as (\ref{eqn:basic_term}), into a series of normal ordered terms, e.g.,
\begin{equation*}
=\sum_{stuvwxyz} \sum_{IJ} \langle I | s^{\dagger}t^{\dagger}u^{\dagger}v^{\dagger}wzyx | J \rangle c_{I} c^{*}_{J} A_{stuvwxyz},
\end{equation*}

\begin{equation*}
+\sum_{stuwxy} \sum_{IJ} \langle I | s^{\dagger}t^{\dagger}u^{\dagger}wzy | J \rangle c_{I} c^{*}_{J} A_{stuwxy},
\end{equation*}

\begin{equation*}
+\sum_{stwx}\sum_{mnop} \sum_{IJ} \langle I | s^{\dagger}t^{\dagger}wz | J \rangle c_{I} c^{*}_{J} A_{stwx},
\end{equation*}

\begin{equation*}
+\sum_{sw}\sum_{mnop} \sum_{IJ} \langle I | s^{\dagger}w | J \rangle c_{I} c^{*}_{J} A_{sw}
\end{equation*}

\begin{equation*}
+\sum_{sw}\sum_{mnop}  A .
\end{equation*}

\noindent Where the tensors $A_{ijkl...}$ are formed by performing the 
contractions between and re-orderings of the indexes of tensors on $Y$ and $Z$ as 
determined from the commutation relations of the creation and annihilation operations, e.g.,
 
\begin{equation*}
A_{stwx} = \sum_{r}^{R}\hat{\wp}_{r}\sum^{c1}_{\{u,y\}}\delta_{uy}\sum^{c2}_{\{v,z\}} \delta_{vz}Y_{ijkl}Z_{mnop},
\end{equation*}

where the $c1$ and $c2$ are sets of pairs of indexes to be contracted, e.g.,
\begin{equation}
c2 = \{ \{i, k\} \} , \{j, k\} \} ,\{l, m\} \}.....\} 
\end{equation}
and $\hat{\wp}_{r}$ which transforms the ordered set of indexes
$\{u,v,s,t,w,x,y,z\}$ into some permutation, $r\in R$, of the ordered set of
indexes $\{i,j,k,l,m,n,o,p\}$, whilst also acting to multiply the result of the
summation by an appropriate factor.\\

\noindent A major advantage of this is that it enables constraints to be
on the values over which the molecular orbital indexes range; all indexes on 
annihilation operators must correspond to orbitals occupied in $|J\rangle$, and 
creation operators must correspond to orbitals occupied in $|I\rangle$. 
This is a consequence of how determinants are constructed in 
active space based methods. A generic determinant can be written 
corresponding to the case where the, molecular orbitals, $\{\psi_{x_{i}}\}$,
with indexes  $X = \{x_{0},...,x_{n}\}$ are occupied can be written
\begin{equation}
\Psi_{K} = \sum^{\zeta \in S_{X}}_{\zeta} \epsilon_{\zeta} \bigotimes_{i=0}^{n}  \psi_{x_{i}},
\end{equation}
where $X$ is a set of molecular orbital indexes $\{x_{0},...,x_{n}\}$,
$\epsilon$ is the nth rank Levi-Cevita
tensor, $\zeta$ is a member of the symmetric group $S_{X}$,
consisting of all possible permutations on the set of indexes $X$.

\noindent In CI based methods it is common to use the fact that 
each distinct ordered set, $X$, of indexes corresponds to a different 
determinant (given all sets being compared have their indexes ordered according
to some standard set of rules). In active space based methods the set of
determinants used to define the wavefunction is constrained to only
those determinants which corresponding $X \in {F^{CAS}}$, where

\begin{equation*}
F^{CAS} = \{ X | (X = X^{c} \cup X^{a}) \wedge (X^{a} \subset P) \wedge ( \text{card}(X) = n_{act} )  \}
\end{equation*}
where $P$ is the list of all "active" molecular orbital indexes, 
$X^{a}$ is a subset of active orbitals occupied in the determinant defined by $X$, and
where $X^{c}$ is the list of closed orbital indexes, all of which are occupied in all determinants defined
from members of $F^{CAS}$.\\

\noindent Unfortunately, $F^{CAS}$ can get very large, leading to computational difficulties. 
One way of trying to manage this is to decompose the set of active orbital indexes, $P$, up into $n_{p}$ subsets;
\begin{equation}
P = \bigcup\limits_{j=0}^{n_{p}} P_{j}, 
\end{equation}
and then splitting $F^{CAS}$ up into subsets, $F_{o_{1},...,o_{n_{p}}}$, based on the 
number, $o_{j}$, of orbitals in each subspace, $P_{j}$, which are occupied : 
\begin{equation}
F_{o_{1},...,o_{n_{p}}} =
\Bigg{\{} X =   X_{c} \cup \Bigg{[} \bigcup\limits_{j=0}^{n_{p}} X_{j}\Bigg{]}
 \Bigg{|} ( X_{j} \subset P_{j} ) \wedge \Bigg{(} \sum_{j}^{n_{p}} o_{j}  = n_{act} \Bigg{)}\Bigg{\}} . 
\end{equation}
Here $X_{j}$ is the subset of $P_{j}$ with cardinality $o_{j}$. The second constraint
is just the requirement that the sum of all the occupancy numbers, $o_{j}$, associated
with each of the different subspaces, $P_{j}$, adds up to the total number of
occupied electrons.\\ 

\noindent To illustrate this consider the case where $P$ is split into two subspaces;
\begin{equation}
P = P_{\mu} \cup  P_{\nu}.
\end{equation}
For the case where
\begin{equation*}
\text{card}(P_{\mu}) = 2 
\text{, \ \ \ }
\text{card}(P_{\nu}) = 4
\text{, \ \ \ and \ \ \ }
n_{act} = 3
\end{equation*}
\noindent The space, $F^{CAS}$, can then be split up according to how many $\gamma$ and $\nu$
electrons are occupied in a given element;
\begin{equation*}
F^{CAS} = F^{CAS}_{3,0} \cup F^{CAS}_{2,1} \cup F^{CAS}_{1,2} 
\end{equation*}
\begin{equation*}
F^{CAS}_{3,0} = \{ X | \wedge (X^{a}_{\mu} \subset P_{mu}) \wedge ( \text{card}(X)_{\mu} = n_{act} \}
\end{equation*}
\begin{equation*}
F^{CAS}_{2,1} = \{ X | \wedge (X^{a}_{\mu} \subset P_{mu}) \wedge ( \text{card}(X)_{\mu} = n_{act}-1 \}
                       \wedge (X^{a}_{\nu} \subset P_{nu}) \wedge ( \text{card}(X)_{\nu} = 1 \}
\end{equation*}
\begin{equation*}
F^{CAS}_{1,2} = \{ X | \wedge (X^{a}_{\mu} \subset P_{\mu}) \wedge ( \text{card}(X)_{\mu} = n_{act}-2 \}
                       \wedge (X^{a}_{\nu} \subset P_{nu}) \wedge ( \text{card}(X)_{\nu} = 2 \}
\end{equation*}
\begin{equation*}
X = X^{c} \cup X^{a}_{\mu} \cup X^{a}_{\nu}.
\end{equation*}
\noindent I shall hereafter refer to these subspaces of $F^{CAS}$ as
ci-sectors (and in specific cases, spin-sectors). Decomposing the active space in
this manner has a number of benefits, as the different ci-sectors may not interact,
or have their interaction limited in some way.
The most prominent example of this is the non-interaction
of ci-sectors with different numbers of occupied $\alpha$ and $\beta$ orbitals 
in the non-relativistic framework.  However, even if they do,
the decomposition of the active space makes it is possible to handle
this interaction a piece wise manner, i.e., deal with interactions
between two ci-sectors at a time.
%\emph{Note : Please correct me if I am wrong,
%but I interpret relativistic CI as a kind of A.S.D., albeit with very different rules 
%governing the interactions between the different sectors. I realize the implementations
%and approaches to optimization are very different.}\\

\noindent One major advantage is that decomposition of the active-space
into different components facilitates the block wise decomposition of the reduced density matrices (RDMs),
as a block of an RDM can being defined by the CI-sectors to 
which the Bra and Ket belong. Using the above decomposition of the 
active space as an example;
\begin{equation*}
\sum_{J}^{ \in \mathcal{F}_{all} }
\sum_{I}^{ \in \mathcal{F}_{all} } \langle I | i^{\dagger}j^{\dagger}m^{\dagger}n^{\dagger}klop | J \rangle c^{M}_{I} c_{J}^{N}
\end{equation*}
\begin{equation*}
\sum^{ \in \mathcal{F}_{all} }_{ \mathcal{F}_{sub} }
\sum^{ \in \mathcal{F}_{all} }_{ \mathcal{F}_{sub} }
\sum^{ \in \mathcal{F}_{sub}}_{I}
\sum^{ \in \mathcal{F}_{sub}}_{J} \langle I | i^{\dagger}j^{\dagger}m^{\dagger}n^{\dagger}klop | J \rangle c^{M}_{I} c_{J}^{N},
\end{equation*}
\noindent where $\mathcal{F}_{sub}$ is one of the sub-spaces
into which the total Fock space $\mathcal{F}_{all}$ has been decomposed.\\

\noindent Unfortunately, this decomposition brings with it a number of
complexities which are not present for undecomposed active spaces. However, 
use of such decomposition techniques can lead to significant gains 
in computational efficiency, particularly for methods which much handle
multi-reference, relativistic wavefunctions.\\ 

\noindent Allowance for the fact that the Bra and Ket may belong to different CI-sectors,
the blockwise decomposition of the molecular orbital tensors, and the exploitation 
of symmetry which exists within and between different blocks \emph{and}  different CI-sectors,
are the distinguishing features of the program.

\section{ Blockwise handling of terms } 
\noindent That different the Bra and Ket may have different CI-sectors has
important ramifications for the program structure, however, before discussing
these in depth it is necessary to discuss the basics of the task list
construction.\\

\noindent In the previous section we discussed the construction of the task list 
used for calculating contractions between multiple different tensors, e.g., the 
task list for obtaining
\begin{equation}
A_{kwzm} = 
[\tilde{B}\tilde{C}\tilde{D}]^{(jy,lx,no,pi)}_{kwzm} = \sum_{jy}\sum_{no}\sum_{lx}\sum_{ip} B_{ijkl}C_{mnop}D_{wxyz} \delta_{jy} \delta_{no} \delta_{lx}\delta_{ip}.
\end{equation}

\noindent  The construction of this "contraction task list" (referred to as \emph{A\_compute\_list} in the code, 
is effectively independent from the task list to be discussed now, which concerns determination of which
contracted tensors, e.g., which $A_{kwzm}$, that need to be calculated in order to obtain the expectation value
or derivative.\\

\noindent As stated in the program overview, the total equation (or expression) to be solved (or evaluated), is 
broken down into a number of \emph{Terms}, e.g., 
\begin{equation*}
\langle M | \hat{Y}\hat{Z} | N \rangle ,
\end{equation*}
which using second quantization as
\begin{equation}
\sum_{ijkl}\sum_{mnop} \sum_{IJ} \langle I | i^{\dagger}j^{\dagger}klm^{\dagger}n^{\dagger}op | J \rangle c^{M}_{I} c_{J}^{N} Y^{ML}_{ijkl}Z^{NP}_{mnop}.
\label{eqn:basic_term_again}
\end{equation}
\noindent In (\ref{eqn:basic_term_again}) the sum over the orbital indexes
$\{i,j,k,l\}$ and  $\{m,n,o,p\}$ runs over all indexes specified by tensors
$\mathbf{Y}$ and $\mathbf{Z}$ respectively. However, we can rewrite the above
in terms of summations over the CI-sectors and blocks into which these tensors may be
decomposed;
\begin{equation}
\sum_{B^{Y}}\sum_{B^{Z}}
\sum^{B^{Y}}_{ijkl}\sum^{B^{Z}}_{mnop} \sum_{IJ} \langle I | i^{\dagger}j^{\dagger}klm^{\dagger}n^{\dagger}op | J \rangle c^{M}_{I} c_{J}^{N} Y^{ML}_{ijkl}Z^{NP}_{mnop}.
\label{eqn:basic_term_block_wise}
\end{equation}
\noindent Here $B^{Y}$ and $B^{Z}$ are blocks of $\mathbf{Y}$ and $\mathbf{Z}$
which define the ranges over which the summations of the molecular orbital
indexes are to occur.\\

\noindent Many of the contributions from these various tensor blocks will
vanish, but exactly which will vanish is dependent on the CI-sector to which
$|I \rangle$ and $|J \rangle$ belong. For example, if 
\begin{equation*}
|I\rangle \in \mathcal{F}_{\mu = 2,\nu = 1} \text{ \ \ \ \ and  \ \ \ \ } |J\rangle \in \mathcal{F}_{\mu = 3,\nu = 0},
\end{equation*}
then the cumulative action of the creation and annihilation operators within
the BraKet must be to destroy one electron in range $r^{\mu}$, and create one
in $r^{\nu}$. All terms corresponding to blocks which do not meet this
criterion can immediately be discarded. This idea can be taken further: If the
expression is rearranged into normal order,
\begin{equation}
\sum_{B^{Y}}\sum_{B^{Z}}
\sum^{B^{Y}}_{ijkl}\sum^{B^{Z}}_{mnop} \sum_{IJ} \langle I | i^{\dagger}j^{\dagger}m^{\dagger}n^{\dagger}klop | J \rangle c^{M}_{I} c_{J}^{N} Y^{ML}_{ijkl}Z^{NP}_{mnop}.
\label{eqn:basic_term_nordered}
\end{equation}
\noindent then not only do we retain the above constraint, but it is also known
that terms corresponding to a combination of blocks, $B^{Y}$ and $B^{Z}$, where
\emph{any} of the indexes $k$, $l$, $o$, $p$ are in range $\nu$, will vanish.
An analogous constraint involving constraints on the ranges of the creation 
operators is obtained by putting the expression into anti-normal ordering;
\begin{equation}
\sum_{B^{Y}}\sum_{B^{Z}}
\sum^{B^{Y}}_{ijkl}\sum^{B^{Z}}_{mnop} \sum_{IJ} \langle I |klop i^{\dagger}j^{\dagger}m^{\dagger}n^{\dagger} | J \rangle c^{M}_{I} c_{J}^{N} Y^{ML}_{ijkl}Z^{NP}_{mnop}.
\label{eqn:basic_term_anordered}
\end{equation}
\noindent By applying this procedure we ensure that only indexes within the
active space are left within the BraKet, which is computationally significant
as the dimension of the active orbital subrange, $r^{a}$, is typically much
smaller than the other subranges, such as those corresponding to core, $r^{c}$, or virtual, $r_{v}$, orbitals.\\

\noindent Whilst this rearrangement will generate new terms in accordance with
the commutation relations of the creation and annihilation operators, all of
these new terms will be of lower rank than the original.  This fact, combined
with the range constraints and elimination of terms means that the increase
number of computational tasks to be performed is generally well worth it.\\  

\noindent Perhaps most importantly of all this means calculations
can be done in a piece-wise fashion, treating only one combination of CI-sectors
at a time. This is particularly useful when calculating expressions requiring reduced density matrix derivatives,
\begin{equation}
\Gamma^{I}_{ijklmn}\sum_{J}\langle I | i^{\dagger}j^{\dagger}m^{\dagger}n^{\dagger}opkl | J \rangle c_{J},
\end{equation}
which due to the lack of a summation over one of the CI-indexes, $I$, can prove extremely large. 
could prove highly problematic for implementations of energy derivative expressions in the
relativistic framework; the treatment of $\alpha$ and $\beta$ electrons effectively doubles
the size of the active space. However, the block decomposition of molecular orbital tensors,
the decomposition of the active space, and the separation of blocks into
a real and imaginary components, ensures that the peak memory requirements of the 
calculations need not exceed those for similar expressions in a non-relativistic framework.\\

\noindent A disadvantage is that the algebraic operations of
rearranging the tensors must be performed for every possible block, of which
there can in principal be millions\footnote{ If each range is divided in to 
six, and we have ten indexes, $6^{10} =60466176 $. }. Fortunately, there
are a number of ways to mitigate this, but it is not so trivial an issue that it can
be completely ignored.\\

\noindent The reordering sequence specified above; normal ordering followed by anti-normal 
ordering, encapsulates the most important principals used by the algebraic manipulator. However,
more efficient task lists can be generated by following these two initial reorderings with others, the
structure of which are determined by the properties of the block, the term being calculated\footnote{
For example, terms involving CI-derivatives use different reorderings to terms involving molecular orbital
tensor derivatives.}. In order to understand why this approach is taken, and how these reorderings
are chosen and generated, it is necessary to discuss the implementation of symmetry.

\section{Handling of block and spin-sector symmetry}

\noindent One of the defining features of the program is that symmetry is
incorporated at the stage of algebraic manipulation, and precedes the task list
construction, i.e., it is only after all expressions have been rewritten to
involve some minimal set of distinct terms that the task list is generated.
Thus if two terms are equivalent, one will of these terms will never appear at
any stage of the final task list.  This is primarily to facilitate the
identification of terms which can be merged or reused (to save memory), and
terms which can be completely separated (to facilitate parallelization).\\

\noindent Whilst the previous section discussed how symmetry was applied in 
execution and representation of the contraction task list, it did not
discuss how symmetry was used to decide what quantities that task list was
attempting to calculated. The focus of this section is the latter of these 
two points.\\

\noindent Broadly speaking, the algorithm should function such that
given the following equivalences:
\begin{equation*}
\hat{T}^{KM}= \hat{T}^{KP} \text{\ , \ \ \  }
\hat{T}^{-1} = T^{\dagger} \text{\ , and \ \ \ }
\mathbf{T}_{b1}^{KM}= \mathbf{T}_{b2}^{KP} 
\end{equation*}
\begin{equation*} 
\mathbf{H}_{b3} = \mathbf{H}_{b4} 
\end{equation*}
\begin{equation*} 
\mathbf{c}_{K}^{P} =  -\boldsymbol{\sigma_{y,2\times 2}}\mathbf{c}_{J}^{M*}, 
|K\rangle =  |\overline{J}\rangle, 
\text{ \ \ \ where  \ \ \ } 
\boldsymbol{\sigma_{y,2\times 2}} = 
\begin{bmatrix}
\boldsymbol{\sigma_{y}} && \mathbf{0} \\ 
\mathbf{0} && \boldsymbol{\sigma_{y}}, \\
\end{bmatrix}
\end{equation*}
\noindent any contribution written using the terms on the left
hand side (LHS) of the above can be rewritten as one involving the terms on the
right hand side (RHS) of the above. For example, 
\begin{equation*}
\sum_{ijkl}^{b1}\sum_{mnop}^{b2} \langle K |
\hat{i}^{\dagger}\hat{j}^{\dagger}\hat{k}\hat{l}\hat{m}^{\dagger}\hat{n}^{\dagger}\hat{o}\hat{p} | K \rangle 
\mathbf{c}_{K}^{P}\mathbf{c}_{I}^{M}T_{ijkl}^{KM,-1,b1}H^{b3}_{mnop}
\end{equation*}
is re-expressed as this term
\begin{equation*}
\sum_{ijkl}^{b1}\sum^{b3}_{mnop}
\langle I |
\hat{m}^{\dagger}\hat{n}^{\dagger}\hat{o}\hat{p} 
\hat{\overline{i}}^{\dagger}\hat{\overline{j}}^{\dagger}\hat{\overline{k}}\hat{\overline{l}}
| \overline{J} \rangle 
\mathbf{c}_{J}^{M*} \mathbf{c}_{I}^{M}
H^{b4}_{mnop}T_{lkji}^{KM,b2}
\end{equation*}
plus some sequence of transformations, and potentially a number of lower order
terms which will arise due to the switching of the order of $\hat{H}$ and
$\hat{T}$. Here the notations $\overline{i}$ is the index corresponding to the
Kramer's pair of orbital $i$. \\

\noindent The key thing to note in the above is that the transformations of the
blocks range limits on the summations are not transformed, whilst those on the
molecular orbital tensors are. Furthermore, the indexes inside the BraKet are
not transformed with the representations of the molecular orbital tensors, but
are transformed due to the time reversal symmetry operation resulting from the
substitution $|K\rangle = |\overline{I}\rangle$, and the Hermitian conjugation
operation applied to $\hat{T}$.\\

\noindent This is because transformations of the operator, such as inversion
and time-reversal, are fundamentally different to transformations of to the
representation of an operator in the molecular orbital basis, e.g., mapping
$T_{ijkl}^{b1} \rightarrow T_{klij}^{b2} $. Transformations to the operator
itself will impact the ranges on which the creation and annihilation operators
used to define that operator act, e.g., in the above case time-reversal alters
the ranges on which the creation and annihilation operators act, and hence
alters the ranges within the BraKet.  Conversely, symmetries arising from
properties of the molecular orbital indexes on the molecular orbital tensors do
not influence the ranges on which the creation and annihilation operators act,
as these are symmetries arising from the molecular orbital basis chosen to
represent the operator.\\

\noindent Clearly, there is a very close link between the two, but they are
distinct. By way of example consider an operator, $\hat{W}$, which is split
into two blocks 
\begin{equation*}
b1 = \{ \alpha,\alpha,\alpha,\alpha \} 
\text { \ \ \ \ and \ \ \ }  
b2 = \{ \beta,\beta,\beta,\beta \} 
\end{equation*}
ranges on indexes. Suppose that ranges $r_{\alpha}$ and $r_{\beta}$ have equal extent, and that 
the representation of $\hat{W}$ within each of these blocks is equivalent, i.e.,
\begin{equation}
W^{b1}_{ijkl}= W^{b2}_{ijkl}.
\end{equation}
Now consider a 2-electron wavefunction, formed from a linear combination of determinants, in each of which 
two alpha orbitals are occupied
\begin{equation}
|M\rangle = \sum_{I} c_{I}^{M} | I (n_{\alpha} = 2 , n_{\beta} = 0 \rangle.
\end{equation}
We can now write
\begin{equation*}
\langle M | \hat{W} |M\rangle = \langle M | \hat{W}^{b1} |M\rangle  + \langle M | \hat{W}^{b2} |M\rangle 
\end{equation*}
\begin{equation*}
= 
\sum_{ijkl}^{b1} \sum_{IJ}\langle I |\hat{i}^{\dagger}\hat{j}^{\dagger}\hat{k}\hat{l} | J \rangle c_{I}c_{J} W^{b1}_{ijkl}+
\sum_{ijkl}^{b2} \sum_{IJ}\langle I |\hat{i}^{\dagger}\hat{j}^{\dagger}\hat{k}\hat{l} | J \rangle c_{I}c_{J} W^{b2}_{ijkl},
\end{equation*}
which then using $W^{b1}_{ijkl}= W^{b2}_{ijkl}$ leads to 
\begin{equation*}
= 
\sum_{ijkl}^{b1} \sum_{IJ}\langle I |\hat{i}^{\dagger}\hat{j}^{\dagger}\hat{k}\hat{l} | J \rangle c_{I}c_{J} W^{b1}_{ijkl}+
\sum_{ijkl}^{b2} \sum_{IJ}\langle I |\hat{i}^{\dagger}\hat{j}^{\dagger}\hat{k}\hat{l} | J \rangle c_{I}c_{J} W^{b1}_{ijkl}.
\end{equation*}
However, it is clear from the definition of $|M\rangle$ that the second term must always be zero, as there are 
no $\beta$ electrons to destroy, hence,
\begin{equation*}
\neq 2 \sum_{ijkl}^{b1} \sum_{IJ}\langle I |\hat{i}^{\dagger}\hat{j}^{\dagger}\hat{k}\hat{l} | J \rangle c_{I}c_{J} W^{b1}_{ijkl}.
\end{equation*}
This means that whilst we can transform all occurrences of $\mathbf{W}^{b2}$
into occurrences of $\mathbf{W}^{b1}$, we are typically unable to apply the
same transformation to the block ranges and the molecular orbital indexes which
appear withing the braket. Transformations of these indexes only arise when
performing transformations on the physical action of the operator, such as
time-reversal, or inversion. \\

\noindent It is for this reason that symmetries are not dealt with by using a
single $i<j<k<l$ type rule, but instead are subdivided into three distinct
types, which can be combined to guide construction of the task list. These
three basic symmetry types are : operator symmetry, representation (or block)
symmetry, and CI-sector symmetry.\\ 

\noindent Operator symmetry refers to transformations on the action of the
operator, and equivalences between operators which act on different
combinations of states (dictated by time reversal, or potentially spatial
symmetry). For example,
\begin{equation}
\hat{T}^{LM} = \hat{T}^{\overline{L}\overline{M}}_{ijkl} 
\end{equation}
\noindent where $\overline{L}$ is the index referring to the Kramers pair of the state with index $L$.
Another kind of operator symmetry is
\begin{equation}
T^{LM, b}_{ijkl} = T^{\overline{L}\overline{M}, \overline{b}}_{ijkl} 
\end{equation}
Representation or block symmetries are relations such as 
\begin{equation}
\mathbf{T}^{LM, b1}_{ijkl} = \rho \mathbf{T}^{LM, b2}_{klij} 
\end{equation}
where $b2$ is some index block which may be obtained by performing a
permutation operation of the indexes of $b1$, and $\rho$ is some multiplicative factor.
Another kind of common block symmetry is
\begin{equation}
\mathbf{T}^{LM, b}_{ijkl} = \rho \mathbf{T}^{\overline{L}\overline{M}, \overline{b}}_{ijkl}* 
\end{equation}
$\overline{b}$ is the range
block defined from the index ranges of the which are Kramers pairs of the
molecular orbital index ranges which define block $b$. Similarly, $\overline{L}$
and  $\overline{M}$ are the indexes of the states which are Kramers pairs of
$L$ and $M$ respectively.\\

\noindent Finally, and example of CI-sector symmetry would be
\begin{equation*} 
\mathbf{c}_{K}^{P} =  -\boldsymbol{\sigma_{y,2\times 2}}\mathbf{c}_{J}^{M*}, 
|K\rangle =  |\overline{J}\rangle, 
\text{ \ \ \ where  \ \ \ } 
\boldsymbol{\sigma_{y,2\times 2}} = 
\begin{bmatrix}
\boldsymbol{\sigma_{y}} && \mathbf{0} \\ 
\mathbf{0} && \boldsymbol{\sigma_{y}} \\
\end{bmatrix}
\end{equation*}
\noindent This is just time reversal symmetry.\\

\noindent Initially, the state dependence of the operators seems peculiar; the
molecular orbitals themselves do not differ between states, hence the state
dependence of the representations suggests that it is not possible to interpret
these operators as describing interactions between 1, 2, or n-electrons.  In
fact, such state dependent operators do not correspond to physical interactions
per se, but are instead a tool used to aid in the description of representation
of a perturbed state in terms via interaction of a number of unperturbed states.
An important consequence of this is that the symmetry of the operator
representations is determined by the form of the equations from which they are
obtained, rather than from consideration of the form of operators themselves.
