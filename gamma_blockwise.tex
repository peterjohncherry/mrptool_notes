\documentclass[12pt]{article}
\usepackage[utf8x]{inputenc}
\usepackage[english]{babel}
\usepackage[T1]{fontenc}
\usepackage{color}
\usepackage{amsmath}
\usepackage{amssymb}
\usepackage{textcomp}
\usepackage{array}
\usepackage{booktabs}
\usepackage[font=small,format=plain,labelfont=bf,up,textfont=it,up]{caption}
\usepackage{longtable}
\usepackage{calc}
\usepackage{setspace}
\usepackage{hhline}
\usepackage{ifthen}
\usepackage{lscape}
\usepackage{pdfpages}
\usepackage{geometry}
 \geometry{
 a4paper,
 total={170mm,257mm},
 left=20mm,
 top=20mm,
 bottom=20mm,
 right=20mm,
 }
\usepackage{cite}
\linespread{1.3}

\begin{document}
\section{Gamma generator}
Consider a term
\begin{equation}
\sum_{ijkl}\sum_{mnop} \sum_{IJ} \langle I | i^{\dagger}j^{\dagger}klm^{\dagger}n^{\dagger}op | J \rangle c^{M}_{I} c_{J}^{N} Y^{ML}_{ijkl}Z^{NP}_{mnop},
\label{eqn:basic_term}
\end{equation}
where $I$ and $J$ each denote a Slater determinant, and $L,M,N,P$ are state
indexes.  $\mathbf{Y}^{ML}$ and $\mathbf{Z}^{NP}$ are representations of
operators and $\hat{Y}$ and $\hat{Z}$ in the molecular orbital basis. For the time being
the state dependence of the operator representations will be ignored, but has important consequences
for exploitation of symmetry, and will be discussed at length later. \\ 

\noindent The proceedure most commonly employed by code generators is to use Wick's theorem
to rearrange expressions, such as (\ref{eqn:basic_term}), into a series of normal ordered terms, e.g.,
\begin{equation*}
=\sum_{stuvwxyz} \sum_{IJ} \langle I | s^{\dagger}t^{\dagger}u^{\dagger}v^{\dagger}wzyx | J \rangle c_{I} c^{*}_{J} A_{stuvwxyz},
\end{equation*}

\begin{equation*}
+\sum_{stuwxy} \sum_{IJ} \langle I | s^{\dagger}t^{\dagger}u^{\dagger}wzy | J \rangle c_{I} c^{*}_{J} A_{stuwxy},
\end{equation*}

\begin{equation*}
+\sum_{stwx}\sum_{mnop} \sum_{IJ} \langle I | s^{\dagger}t^{\dagger}wz | J \rangle c_{I} c^{*}_{J} A_{stwx},
\end{equation*}

\begin{equation*}
+\sum_{sw}\sum_{mnop} \sum_{IJ} \langle I | s^{\dagger}w | J \rangle c_{I} c^{*}_{J} A_{sw}
\end{equation*}

\begin{equation*}
+\sum_{sw}\sum_{mnop}  A .
\end{equation*}

\noindent Where the tensors $A_{ijkl...}$ are formed by performing the 
contractions between and reorderings of the indexes of tensors on $Y$ and $Z$ as 
determined from the commutation relations of the creation and annihilation operations, e.g.,
 
\begin{equation*}
A_{stwx} = \sum_{r}^{R}\hat{\wp}_{r}\sum^{c1}_{\{u,y\}}\delta{uy}\sum^{c2}_{\{v,z\}} \delta{vz}Y_{ijkl}Z_{mnop},
\end{equation*}

where the $c1$ and $c2$ are sets of pairs of indexes to be contracted, e.g.,
\begin{equation}
c2 = \{ \{i, k\} \} , \{j, k\} \} ,\{l, m\} \}.....\} 
\end{equation}
and $\hat{\wp}_{r}$ which transforms the ordered set of indexes
$\{u,v,s,t,w,x,y,z\}$ into some permutation, $r\in R$, of the ordered set of
indexes $\{i,j,k,l,m,n,o,p\}$, whilst also acting to multiply the result of the
summation by an appropriate factor.\\

\noindent A major advantage of this is that it enables constraints to be
on the values over which the molecular orbital indexes; all indexes on 
annihilation operators must correspond to orbitals occupied in $|J\rangle$, and 
creation operators must correspond to orbitals occupied in $|I\rangle$. 
This is a consequence of how determinants are constructed in 
active space based methods. A generic determinant can be written
\begin{equation}
\Psi_{K} = \sum^{p\in S_{X}}_{p} \epsilon_{x} \bigotimes_{i=0}^{n}  \psi_{x_{i}},
\end{equation}
where $X$ is a set of molecular orbital indexes $\{x_{0},...,x_{n}\}$,
$\epsilon$ is the nth rank Levi-Cevita
tensor, $p$ is a member of the symmetric group $S_{X}$,
consisting of all possible permutations on the set of indexes,
$X = \{x_{0},...,x_{n}\}$, of molecular orbitals $\psi_{x_{i}}$ from which the
determinant is constructed.\\

\noindent In CI based methods it is common to use the fact that 
each distinct ordered set, $X$, of indexes corresponds to a different 
determinant (given all sets being compared have their indexes ordered according
to some standard set of rules). In active space based methods number of
determinants used to define the wavefunction is constrained to only
those determinants which corresponding to sets $X \in {F^{CAS}}$, where

\begin{equation*}
F^{CAS} = \{ X | (X = X^{c} \cup X^{a}) \wedge (X^{a} \subset P) \wedge ( \text{card}(X) = n_{act} )  \}
\end{equation*}
where $P$ is the list of all "active" molecular orbital indexes, 
$X^{a}$ is a subset of set of $n_{act}$ orbitals occupied in the determinant defined by $X$, and
where $X^{c}$ is the list of closed orbital indexes, all of which are occupied in all determinants.\\

\noindent Unfortunately, $F^{CAS}$ can get very large, leading to computational difficulties. 
One way of trying to manage this is to decompose the set of active orbital indexes, $P$, up into subsets;
\begin{equation}
P = \bigcup\limits_{j=0} P_{j}, 
\end{equation}
and performing a similar decomposition on the sets of molecular orbital indexes $X$;
\begin{equation}
X = X_{c} \cup  \bigcup\limits_{j=0} X_{j} \text{ \ \ \ \ \ where \ \ \ \ \ } X_{j} \subset P_{j}. 
\end{equation}
This leads to a related decomposition of $F^{CAS}$ into smaller subspaces;
\begin{equation*}
F^{CAS} = \cup_{n_{1},..,n_{N}} F^{CAS}_{p_{1},...,p_{N}},
\end{equation*}
where the indexes ${p_{1},...,p_{N}}$, define the number of elements any given $X \in F^{CAS}_{p_{1},...,p_{N}}$
may have in common with each of the different subsets of $P$.\\

\noindent  For example, suppose $P$ is split into $\mu$ and $\nu$ indexes;
\begin{equation}
P = P_{\mu} \cup  P_{\nu}.
\end{equation}
For the case where
\begin{equation*}
\text{card}(P_{\mu}) = 2 
\text{, \ \ \ }
\text{card}(P_{\nu}) = 4
\text{, \ \ \ and \ \ \ }
n_{act} = 3
\end{equation*}
The space, $F^{CAS}$, can then be split up according to how many $\gamma$ and $\nu$
electrons are occupied in a given element;
\begin{equation*}
F^{CAS} = F^{CAS}_{3,0} \cup F^{CAS}_{2,1} \cup F^{CAS}_{1,2} 
\end{equation*}
\begin{equation*}
F^{CAS}_{3,0} = \{ X | \wedge (X^{a}_{\mu} \subset P_{mu}) \wedge ( \text{card}(X)_{\mu} = n_{act} \}
\end{equation*}
\begin{equation*}
F^{CAS}_{2,1} = \{ X | \wedge (X^{a}_{\mu} \subset P_{mu}) \wedge ( \text{card}(X)_{\mu} = n_{act}-1 \}
                       \wedge (X^{a}_{\nu} \subset P_{nu}) \wedge ( \text{card}(X)_{\nu} = 1 \}
\end{equation*}
\begin{equation*}
F^{CAS}_{1,2} = \{ X | \wedge (X^{a}_{\mu} \subset P_{\mu}) \wedge ( \text{card}(X)_{\mu} = n_{act}-2 \}
                       \wedge (X^{a}_{\nu} \subset P_{nu}) \wedge ( \text{card}(X)_{\nu} = 2 \}
\end{equation*}
\begin{equation*}
X = X^{c} \cup X^{a}_{\mu} \cup X^{a}_{\nu}.
\end{equation*}
\noindent I shall hereafter refer to these subspaces of $F^{CAS}$ as
ci-sectors (and in specific cases, spin-sectors). Decomposing the active space in
this manner has a number of benefits, as the different ci-sectors may not interact,
the most prominent example of this is the non-interaction
of ci-sectors with different numbers of occupied $\alpha$ and $\beta$ orbitals 
in the non-relativistic framework.  However, even if they do,
the decomposition of the active space makes it is possible to handle
this interaction a piece wise manner, i.e., deal with interactions
between two ci-sectors at a time. \emph{Note : Please correct me if I am wrong,
but I interpret relativistic CI as a kind of A.S.D., albeit with very different rules 
governing the interactions between the different sectors. I realize the implementations
and approaches to optimization are very different.}\\

\noindent One major advantage is that decomposition of the active-space
into different components faciliates the block wise decomposition of the reduced density matrices (RDMs),
as a block of an RDM can being defined by the CI-sectors to 
which the Bra and Ket belong. Using the above decomposition of the 
active space as an example;
\begin{equation*}
\sum_{J}^{ \in \mathcal{F}_{all} }
\sum_{I}^{ \in \mathcal{F}_{all} } \langle I | i^{\dagger}j^{\dagger}m^{\dagger}n^{\dagger}klop | J \rangle c^{M}_{I} c_{J}^{N}
\end{equation*}
\begin{equation*}
\sum^{ \in \mathcal{F}_{all} }_{ \mathcal{F}_{sub} }
\sum^{ \in \mathcal{F}_{all} }_{ \mathcal{F}_{sub} }
\sum^{ \in \mathcal{F}_{sub}}_{I}
\sum^{ \in \mathcal{F}_{sub}}_{J} \langle I | i^{\dagger}j^{\dagger}m^{\dagger}n^{\dagger}klop | J \rangle c^{M}_{I} c_{J}^{N},
\end{equation*}
\noindent where $\mathcal{F}_{sub}$ is one of the sub-spaces into which the total Fock space $\mathcal{F}_{all}$ has been decomposed.

\noindent Unfortunately, this decomposition brings with it a number of
complexities which are not present for undecomposed active spaces. However, 
as we shall hopefully see, the reduction in the size, number, and complexity is
substanstial, and is particularly valuable when treating multi-reference,
relativistic wavefunctions.\\ 

\noindent Allowance for the fact that Bra and Ket may belong to different CI-sectors,
the blockwise decomposition of the molecular orbital tensors, and the exploitation 
of symmetry which exists within and between different blocks \emph{and}  different CI-sectors,
are the distinguishing features of the program.

\subsection{ Blockwise handling of terms } 
\noindent That different the Bra and Ket may have different CI-sectors has
important ramifications for the program structure, however, before discussing
these in depth it is necessary to discuss the basics of the task list
construction.\\

\noindent As discussed in section(), we typically wish to evaluate terms of the
following form;
\begin{equation}
\sum_{ijkl}\sum_{mnop} \sum_{IJ} \langle I | i^{\dagger}j^{\dagger}klm^{\dagger}n^{\dagger}op | J \rangle c^{M}_{I} c_{J}^{N} Y^{ML}_{ijkl}Z^{NP}_{mnop}.
\label{eqn:basic_term_again}
\end{equation}
\noindent In (\ref{eqn:basic_term_again}) the sum over the orbital indexes
$\{i,j,k,l\}$ and  $\{m,n,o,p\}$ runs over all indexes specified by tensors
$\mathbf{Y}$ and $\mathbf{Z}$ respectively. However, we can rewrite the above
in terms of summations over the ci-sectors and blocks into which these tensors may be
decomposed;
\begin{equation}
\sum_{B^{Y}}\sum_{B^{Z}}
\sum^{B^{Y}}_{ijkl}\sum^{B^{Z}}_{mnop} \sum_{IJ} \langle I | i^{\dagger}j^{\dagger}klm^{\dagger}n^{\dagger}op | J \rangle c^{M}_{I} c_{J}^{N} Y^{ML}_{ijkl}Z^{NP}_{mnop}.
\label{eqn:basic_term_block_wise}
\end{equation}
\noindent Here $B^{Y}$ and $B^{Z}$ are blocks of $\mathbf{Y}$ and $\mathbf{Z}$
which define the ranges over which the summations of the molecular orbital
indexes are to occur.\\

\noindent Many of the contributions from these various tensor blocks will
vanish, but exactly which will vanish is dependent on the CI-sector to which
$|I \rangle$ and $|J \rangle$ belong. For example, if 
\begin{equation*}
|I\rangle \in \mathcal{F}_{\mu = 2,\nu = 1} \text{ \ \ \ \ and  \ \ \ \ } |J\rangle \in \mathcal{F}_{\mu = 3,\nu = 0},
\end{equation*}
then the cumulative action of the creation and annihilation operators within
the BraKet must be to destroy one electron in range $r^{\mu}$, and create one
in $r^{\nu}$. All terms corresponding to blocks which do not meet this
criterion can immediately be discarded. This idea can be taken further: If the
expression is rearranged into normal order,
\begin{equation}
\sum_{B^{Y}}\sum_{B^{Z}}
\sum^{B^{Y}}_{ijkl}\sum^{B^{Z}}_{mnop} \sum_{IJ} \langle I | i^{\dagger}j^{\dagger}m^{\dagger}n^{\dagger}klop | J \rangle c^{M}_{I} c_{J}^{N} Y^{ML}_{ijkl}Z^{NP}_{mnop}.
\label{eqn:basic_term_nordered}
\end{equation}
\noindent then not only do we retain the above constraint, it is also known
that terms corresponding to a combination of blocks, $B^{Y}$ and $B^{Z}$, where
\emph{any} of the indexes $k$, $l$, $o$, $p$ are in range $\nu$, will vanish.
An analagous constraint involing constraints on the ranges of the creation 
opeators is obtained by putting the expression into anti-normal ordering;
\begin{equation}
\sum_{B^{Y}}\sum_{B^{Z}}
\sum^{B^{Y}}_{ijkl}\sum^{B^{Z}}_{mnop} \sum_{IJ} \langle I |klop i^{\dagger}j^{\dagger}m^{\dagger}n^{\dagger} | J \rangle c^{M}_{I} c_{J}^{N} Y^{ML}_{ijkl}Z^{NP}_{mnop}.
\label{eqn:basic_term_anordered}
\end{equation}
\noindent By applying this proceedure we ensure that only indexes within the
active space are left within the BraKet, which is computationally significant
as the dimension of the active orbitals range, $r^{a}$, is typically much
smaller than the other blocks or the molecular orbital tensors (such as the
core, $r^{c}$, or virtual $r_{c}$ blocks).\\

\noindent Whilst this rearrangement will generate new terms in accordance with
the commutation relations of the creation and annihilation operators, all of
these new terms will be of lower rank than the original.  This fact, combined
with the range constraints and elimination of terms means that the increase
number of computational tasks to be performed is generally well worth it.\\  

\noindent Perhaps most importantly of all is that it means that
all calculations can be done piece-wise, treating only one combination of CI-sectors
at a time. This is particularly useful when calculating expressions requiring reduced density matrix derivatives;
\begin{equation}
\Gamma^{I}_{ijklmn}\sum_{J}\langle I | i^{\dagger}j^{\dagger}m^{\dagger}n^{\dagger}opkl | J \rangle c_{J},
\end{equation}
which due to the lack of a summation over one of the CI-indexes, $I$, can prove extremely large. This
would prove highly problematic for implementations of energy derivative expressions in the
relativistic framework. However, the block decomposition of molecular orbital tensors,
the decomposition of the active space, and the seperation of blocks into
a real and imaginary components, ensures that the peak memory requirements of the 
calculations need not exceed those for similar expressions in a non-relativstic framework.\\

\noindent A disadvantage is that the algebraic operations of
rearranging the tensors must be performed for every possible block, of which
there can in principal be millions\footnote{ If each range is divided in to 
six, and we have ten indexes, then $6^{10} =60466176 $. }. Fortunately, there
are a number of ways to mitigate this, but it is not so trivial an issue that it can
be completely ignored.\\

\noindent The reordering sequence specified above; normal ordering followed by anti-normal 
ordering, encapsulates the most important principals used by the algebraic manipulator. However,
more efficient task lists can be generated by following these with other reorderings, whose structure
is determined by the properties of the block, and ultimately, the term being calculated. In order to understand
why this approach is taken, and how these reorderings are chosen and generated, it is necessary to discuss the implementation
of symmetry.

\section{Handling of Block Symmetry}

\noindent Initially, the state dependence of the operators seems perculiar; the
molecular orbitals themselves do not differ between states, hence the state
dependence of the representations suggests that it is not possible to interpret
these operators as describing interactions between 1, 2, or n-electrons.  In
fact, such state dependent operators do not correspond to physical interactions
per se, but are instead a tool used to aid in the description of representation
of a perturbed state interms via interaction of a number of unperturbed states.
An important consequence of this is that the symmetry of the operator
representations is determined by the form of the equations from which they are
obtained, rather than from consideration of the form of operators themselves.

\end{document}
