\documentclass[12pt]{article}
\usepackage[utf8x]{inputenc}
\usepackage[english]{babel}
\usepackage[T1]{fontenc}
\usepackage{color}
\usepackage{amsmath}
\usepackage{amssymb}
\usepackage{textcomp}
\usepackage{array}
\usepackage{booktabs}
\usepackage[font=small,format=plain,labelfont=bf,up,textfont=it,up]{caption}
\usepackage{longtable}
\usepackage{calc}
\usepackage{setspace}
\usepackage{hhline}
\usepackage{ifthen}
\usepackage{lscape}
\usepackage{pdfpages}
\usepackage{geometry}
 \geometry{
 a4paper,
 total={170mm,257mm},
 left=20mm,
 top=20mm,
 bottom=20mm,
 right=20mm,
 }
\usepackage{cite}
\linespread{1.3}

\begin{document}
\section{Gamma generator}
Consider a term
\begin{equation}
\sum_{ijkl}\sum_{mnop} \sum_{IJ} \langle I | i^{\dagger}j^{\dagger}klm^{\dagger}n^{\dagger}op | J \rangle c^{M}_{I} c_{J}^{N} Y^{ML}_{ijkl}Z^{NP}_{mnop},
\label{eqn:basic_term}
\end{equation}
where $I$ and $J$ each denote a Slater determinant, and $L,M,N,P$ are state
indexes.  $\mathbf{Y}^{ML}$ and $\mathbf{Z}^{NP}$ are representations of
operators and $\hat{Y}$ and $\hat{Z}$ in the molecular orbital basis. For the time being
the state dependence of the operator representations will be ignored, but has important consequences
for exploitation of symmetry, and will be discussed at length later. \\ 

\noindent The proceedure most commonly employed by code generators is to use Wick's theorem
to rearrange expressions, such as (\ref{eqn:basic_term}), into a series of normal ordered terms, e.g.,
\begin{equation*}
=\sum_{stuvwxyz} \sum_{IJ} \langle I | s^{\dagger}t^{\dagger}u^{\dagger}v^{\dagger}wzyx | J \rangle c_{I} c^{*}_{J} A_{stuvwxyz},
\end{equation*}

\begin{equation*}
+\sum_{stuwxy} \sum_{IJ} \langle I | s^{\dagger}t^{\dagger}u^{\dagger}wzy | J \rangle c_{I} c^{*}_{J} A_{stuwxy},
\end{equation*}

\begin{equation*}
+\sum_{stwx}\sum_{mnop} \sum_{IJ} \langle I | s^{\dagger}t^{\dagger}wz | J \rangle c_{I} c^{*}_{J} A_{stwx},
\end{equation*}

\begin{equation*}
+\sum_{sw}\sum_{mnop} \sum_{IJ} \langle I | s^{\dagger}w | J \rangle c_{I} c^{*}_{J} A_{sw}
\end{equation*}

\begin{equation*}
+\sum_{sw}\sum_{mnop}  A .
\end{equation*}

\noindent Where the tensors $A_{ijkl...}$ are formed by performing the 
contractions between and reorderings of the indexes of tensors on $Y$ and $Z$ as 
determined from the commutation relations of the creation and annihilation operations, e.g.,
 
\begin{equation*}
A_{stwx} = \sum_{r}^{R}\hat{\wp}_{r}\sum^{c1}_{\{u,y\}}\delta{uy}\sum^{c2}_{\{v,z\}} \delta{vz}Y_{ijkl}Z_{mnop},
\end{equation*}

where the $c1$ and $c2$ are sets of pairs of indexes to be contracted, e.g.,
\begin{equation}
c2 = \{ \{i, k\} \} , \{j, k\} \} ,\{l, m\} \}.....\} 
\end{equation}
and $\hat{\wp}_{r}$ which transforms the ordered set of indexes
$\{u,v,s,t,w,x,y,z\}$ into some permutation, $r\in R$, of the ordered set of
indexes $\{i,j,k,l,m,n,o,p\}$, whilst also acting to multiply the result of the
summation by an appropriate factor.\\

\noindent A major advantage of this is that it enables constraints to be
on the values over which the molecular orbital indexes; all indexes on 
annihilation operators must correspond to orbitals occupied in $|J\rangle$, and 
creation operators must correspond to orbitals occupied in $|I\rangle$.\\

\noindent This is due to properties of how the determinants are constructed in 
active space based methods. A generic determinant is   
\begin{equation}
\Psi_{K} = \sum^{p\in S_{X}}_{p} \epsilon_{x} \bigotimes_{i=0}^{n}  \psi_{x_{i}}
\end{equation}
where $X$ is a set of molecular orbital indexes $\{x_{0},...,x_{n}\}$,
$\epsilon$ is the nth rank Levi-Cevita
tensor, $p$ is a member of the symmetric group $S_{X}$,
consisting of all possible permutations on the set of indexes,
$X = \{x_{0},...,x_{n}\}$, of molecular orbitals $\psi_{x_{i}}$ from which the
determinant is constructed.\\


\noindent In CI based methods it is common to use the fact that 
each distinct ordered set, $X$, of indexes corresponds to a different 
determinant (given all sets being compared have their indexes ordered according
to some standard set of rules). In active space based methods number of
determinants used to define the wavefunction is constrained to only
those determinants which corresponding to sets $X \in {F^{CAS}}$, where

\begin{equation*}
F^{CAS} = \{ X | (X = X^{c} \cup X^{a}) \wedge (X^{a} \subset P) \wedge ( \text{card}(X) = n_{act} )  \}
\end{equation*}
where $P$ is the list of all active molecular orbital indexes, 
$X^{a}$ is a subset of set of $n_{act}$ orbitals occupied in the determinant defined by $X$, and
where $X^{c}$ is the list of closed orbital indexes, all of which are occupied in all determinants.\\

\noindent Unfortunately, $F^{CAS}$ can get very large, leading to computational difficulties. 
One way of trying to manage this is to decompose the set of active orbital indexes, $P$, up into subsets;
\begin{equation}
P = \cup_{j} P_{j}, 
\end{equation}
and performing a similar decomposition on the sets of molecular orbital indexes $X$;
\begin{equation}
X = (\cup_{j} X_{j}) \cup X_{c} \text{ \ \ \ \ \ where \ \ \ \ \ } X_{j} \subset P_{j}. 
\end{equation}
This leads to a related decomposition of $F^{CAS}$ into smaller subspaces;
\begin{equation*}
F^{CAS} = \cup_{n_{1},..,n_{N}} F^{CAS}_{p_{1},...,p_{N}},
\end{equation*}
where the indexes ${p_{1},...,p_{N}}$, define the number of elements any given $X \in F^{CAS}_{p_{1},...,p_{N}}$
may have in common with each of the different subsets of $P$.\\

\noindent  For example, suppose $P$ is split into $\mu$ and $\nu$ indexes;
\begin{equation}
P = P_{\mu} \cup  P_{\nu}.
\end{equation}
For the case where
\begin{equation*}
\text{card}(P_{\mu}) = 2 
\text{, \ \ \ }
\text{card}(P_{\nu}) = 4
\text{, \ \ \ and \ \ \ }
n_{act} = 3
\end{equation*}
The space, $F^{CAS}$, can then be split up according to how many $\gamma$ and $\nu$
electrons are occupied in a given element;
\begin{equation*}
F^{CAS} = F^{CAS}_{3,0} \cup F^{CAS}_{2,1} \cup F^{CAS}_{1,2} 
\end{equation*}
\begin{equation*}
F^{CAS}_{3,0} = \{ X | \wedge (X^{a}_{\mu} \subset P_{mu}) \wedge ( \text{card}(X)_{\mu} = n_{act} \}
\end{equation*}
\begin{equation*}
F^{CAS}_{2,1} = \{ X | \wedge (X^{a}_{\mu} \subset P_{mu}) \wedge ( \text{card}(X)_{\mu} = n_{act}-1 \}
                                              \wedge (X^{a}_{\nu} \subset P_{nu}) \wedge ( \text{card}(X)_{\nu} = 1 \}
\end{equation*}
\begin{equation*}
F^{CAS}_{1,2} = \{ X | \wedge (X^{a}_{\mu} \subset P_{mu}) \wedge ( \text{card}(X)_{\mu} = n_{act}-2 \}
                                              \wedge (X^{a}_{\nu} \subset P_{nu}) \wedge ( \text{card}(X)_{\nu} = 2 \}
\end{equation*}
\begin{equation*}
X = X^{c} \cup X^{a}_{mu} \cup X^{a}_{\nu}.
\end{equation*}
\noindent I shall hereafter refer to these subspaces of $F^{CAS}$ as
ci-sectors, or sometimes spin-sectors. Decomposing the active space in
this manner has a number of benefits, as the different ci-sectors may not interact,
the most prominent example of this is the non-interaction
of ci-sectors with different numbers of occupied $\alpha$ and $\beta$ orbitals 
in the non-relativistic framework.  However, even if they do,
the decomposition of the active space makes it is possible to handle
this interaction a piece wise manner, i.e., deal with interactions
between two ci-sectors at a time.\\

\noindent Unfortunately, this decomposition brings with it a number of
complexities which are not present for undecomposed active spaces. However, 
as we shall hopefully see, the reduction in the size, number, and complexity is
substanstial, and is imperative to efficient handling of the relativistic wavefunctions.\\

\noindent Initially, the state dependence of the operators seems perculiar; the
molecular orbitals themselves do not differ between states, hence the state
dependence of the representations suggests that it is not possible to interpret
these operators as describing interactions between 1, 2, or n-electrons.  In
fact, such state dependent operators do not correspond to physical interactions
per se, but are instead a tool used to aid in the description of representation
of a perturbed state interms via interaction of a number of unperturbed states.
An important consequence of this is that the symmetry of the operator
representations is determined by the form of the equations from which they are
obtained, rather than from consideration of the form of operators themselves.

\end{document}
