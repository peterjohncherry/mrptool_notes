\documentclass[12pt]{article}
\usepackage[utf8x]{inputenc}
\usepackage[english]{babel}
\usepackage[T1]{fontenc}
\usepackage{color}
\usepackage{amsmath}
\usepackage{amssymb}
\usepackage{textcomp}
\usepackage{array}
\usepackage{booktabs}
\usepackage[font=small,format=plain,labelfont=bf,up,textfont=it,up]{caption}
\usepackage{longtable}
\usepackage{calc}
\usepackage{setspace}
\usepackage{hhline}
\usepackage{ifthen}
\usepackage{lscape}
\usepackage{pdfpages}
\usepackage{geometry}
 \geometry{
 a4paper,
 total={170mm,257mm},
 left=20mm,
 top=20mm,
 bottom=20mm,
 right=20mm,
 }
\usepackage{cite}
\linespread{1.3}

\begin{document}
Have perturbed Hamiltonian:
\begin{equation}
\hat{H}\Psi = ( H_{0} +V )( \psi_{0}+ \phi ) = E \Psi = (\epsilon_{0} + \Delta \epsilon) \Psi,
\end{equation}
and require that 
\begin{equation}
H_{0}\psi_{0} = \epsilon_{0}\psi_{0}.
\end{equation}
\noindent Have the normalization conditions 
\begin{equation}
\langle \Psi | \psi_{0} \rangle = 1 \text{ \ \ \ \ } 
\langle \psi_{0} | \psi_{0} \rangle = 1 \text{ \ \ \ \ } 
\langle \phi | \psi_{0} \rangle = 1 
\end{equation}
\noindent In this case our perturbation excites is essentially splitting the Hailtonian into 
two parts. The Hamiltonian is a bilinear operator. For $\hat{H}_{0}$ we
say both arguments must be within the CASCI space, whereas for $\hat{V}$, one argument must be in 
the external space. Hence,
\begin{equation}
\langle \psi_{0} | \hat{V} | \psi_{0} \rangle = \langle \hat{V} \rangle = 0
\end{equation}
Now define a resolvent operator, $R(z)$;
\begin{equation}
R(z) = (z\hat{I}-\hat{H})^{-1}
\end{equation} 
here, $\hat{I}$ is the identity. Generally speaking, $z$ can be any complex number, but it is useful
to restrict ourselves to cases where $\Re e(z) = E$. This means we can then write $R(z)= R(E+i\eta) $. 
It is useful to note that
\begin{equation}
R(E+i\eta)(E+i\eta-H)= 1
\end{equation}
\begin{equation}
\lim_{\eta \rightarrow 0_{+}} R(E+i\eta)(E-H)= 1
\end{equation}
\noindent This resolvent operator is can help us to rearrange the Schrodinger equation to get an expression for
$\phi$
\begin{equation*}
\hat{H} (|\psi_{0}\rangle + |\phi\rangle)  = E(|\psi\rangle+ |\phi\rangle)
\end{equation*}
\begin{equation*}
 (E-\hat{H}) | \phi\rangle
=
(\hat{H}-E)|\psi_{0} \rangle
 \end{equation*}
\begin{equation*}
|\phi \rangle =  (E-\hat{H})^{-1} (\hat{H}-E)|\psi_{0} \rangle
\end{equation*}
\begin{equation*}
 |\phi \rangle = R(E + i\eta) (\hat{H}-E)|\psi_{0} \rangle 
\end{equation*}
\begin{equation}
 |\phi \rangle = R(E + i\eta) (\hat{V}-\delta E)|\psi_{0} \rangle
\end{equation}
We can use this definition of $\phi$ to write the full wavefunction $\Psi$ in
terms of the unperturbed wavefunction
\begin{equation}
\Psi =  (1+ R(E + i\eta) )(\hat{V}-\delta E))|\psi_{0} \rangle
\end{equation}
Using the inverse properties of the resolvent mentioned earlier
\begin{equation}
\Psi = \hat{Q}|\psi_{0}\rangle = \lim_{\eta \rightarrow 0_{+}}i\eta R(E+i\eta)|\psi_{0}\rangle
\end{equation}
This can then be used to write
\begin{equation}
(E-\hat{H}) \hat{Q}\psi_{0}  = 0 
\label{eqn:proj} 
\end{equation}
We now approximate $\hat{Q}$ as 
\begin{equation}
\hat{Q} = 1 + \hat{\hat{T}} = 1 + \sum_{ijkl}a_{i}^{\dagger}a^{\dagger}_{j}a_{k}a_{l}T_{ijkl}
\end{equation}
where $T_{ijkl}$ is a coefficient associated with a given combination $\{i,j,k,l\}$ of orbital excitations. Under this
approximation 

\begin{equation}
\langle\phi|(E-\hat{H}_{0}+\hat{V}) \hat{Q}|\psi_{0}\rangle  = 0 
\label{eqn:proj} 
\end{equation}
\begin{equation}
E  = 
\langle\phi|(\hat{H}_{0}+\hat{V}) \hat{Q}|\psi_{0}\rangle + 
\langle\phi|(\hat{H}_{0}+\hat{V}) \hat{Q}|\psi_{0}\rangle + 
\langle\phi|(\hat{H}_{0}+\hat{V}) \hat{Q}|\psi_{0}\rangle  = 0 
\label{eqn:proj} 






 To acheive this first use the the intermediate normalization conditions
to write 
\begin{equation}
E = \langle \psi_{0} | \hat{H}_{0} + \hat{V}  |  \psi_{0} + \phi \rangle  = E_{0} + \langle \psi_{0} | \hat{V} | \Psi \rangle,
\end{equation}
subsituting from (\ref{eqn:proj}) we can write this terms of $\psi_{0}$ 
\begin{equation}
E = \langle \psi_{0} | \hat{H}_{0} + \hat{V}\hat{Q}|  \psi_{0} \rangle 
\end{equation}
So the perturbation energy is 
\begin{equation}
\Delta E = \langle \psi_{0} | H | \phi \rangle + \langle \psi_{0} | \hat{V}\hat{Q}|  \psi_{0} \rangle . 
\label{eqn:pt_energy_Q}
\end{equation}

\begin{equation}
|\Psi \rangle \approx |\psi_{0} \rangle + \sum_{ijkl} \hat{T}_{ijkl} |\psi_{0} \rangle,
\end{equation}
\begin{equation}
\text{\ \ \ i.e. \ \ \ } | \phi \rangle \approx \sum_{ijkl} \hat{T}_{ijkl} |\psi_{0} \rangle 
\end{equation}
Substituing this into (\ref{eqn:pt_energy_Q}) yields 
\begin{equation}
\Delta E = \langle \psi_{0} | H \hat{T}| \psi_{0} \rangle + \langle \psi_{0} | \hat{V}\hat{T}| \psi_{0} \rangle . 
\label{eqn:pt_energy_T}
\end{equation}
This can be used to obtain the and expression for the perturbation energy.
%\begin{equation}
%= \langle \psi_{0} | H | \phi \rangle +\lim_{\eta \rightarrow 0_{+}} \langle \psi_{0} | \hat{V} (i\eta R(E+i\eta))  |\psi_{0} \rangle . 
%\label{eqn:init_res}
%\end{equation}
%The resolvent is often represented as a transform on the wavefunctions 
%\begin{equation}
%R(E) = \frac{1}{\hat{H}-E} = \sum_{N} \frac{|\Psi_{N}\rangle \langle \Psi_{N}| }{\langle \Psi_{N}| \hat{H}-E  | \Psi_{N} \rangle},
%\end{equation}
%where $\{\Psi \}$ are the eigenvectors of $\hat{H}$. Substituting this into (\ref{eqn:init_res}), and omitting the $\eta$ for clarity, yields
%\begin{equation*}
%\Delta E = \langle \psi_{0} | H | \phi \rangle + \sum_{N} \frac{\langle \psi_{0} | \hat{V}  |\Psi_{N}\rangle \langle \Psi_{N}|\psi_{0} \rangle }{\langle \Psi_{N}| \hat{H}-E  | \Psi_{N} \rangle}. 
%\end{equation*}
%\begin{equation}
%= \langle \psi_{0} | H | \phi \rangle + \sum_{N} \frac{\langle \psi_{0} | \hat{V}  |\phi_{N}\rangle  }{\langle \phi_{N}| \hat{H}-E  | \phi_{N} \rangle}. 
%\end{equation}


%but this requires that 
%\begin{equation}
%\hat{H} = \sum_{N=1}^{\infty} E_{N} | N \rangle \langle N | ,
%\end{equation}
%i.e., that the states $\{ N \}$ are eigenvalues of $\hat{H}$. This is not the case for our Hamiltonian. Consequently, 
%we approximate the Hamiltonian with $\hat{f}^{sa}$,  the state-averaged Fock operator, which is a single particle operator.

\end{document}
